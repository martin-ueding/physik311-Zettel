% Copyright © 2012 Martin Ueding <dev@martin-ueding.de>
%
\documentclass[11pt, ngerman, fleqn]{article}

\usepackage[a4paper, left=3cm, right=2cm, top=2cm, bottom=2cm]{geometry}
\usepackage[activate]{pdfcprot}
\usepackage[cdot, squaren]{SIunits}
\usepackage[iso]{isodate}
\usepackage[parfill]{parskip}
\usepackage[T1]{fontenc}
\usepackage[utf8]{inputenc}
\usepackage{amsmath}
\usepackage{amssymb}
\usepackage{amsthm}
\usepackage{babel}
\usepackage{color}
\usepackage{commath}
\usepackage{epstopdf}
\usepackage{fancyhdr}
\usepackage{graphicx}
\usepackage{hyperref}
\usepackage{setspace}
\usepackage{tikz}

\usepackage[charter]{mathdesign}

\definecolor{darkblue}{rgb}{0,0,.5}
\definecolor{darkgreen}{rgb}{0,.5,0}

\hypersetup{
	breaklinks=false,
	citecolor=darkgreen,
	colorlinks=true,
	linkcolor=black,
	menucolor=black,
	urlcolor=darkblue,
}

\setlength{\columnsep}{1.5cm}

\DeclareMathOperator{\arcsinh}{arsinh}
\DeclareMathOperator{\arsinh}{arsinh}
\DeclareMathOperator{\asinh}{arsinh}
\DeclareMathOperator{\card}{card}
\DeclareMathOperator{\diam}{diam}

\newcommand{\dalambert}{\mathop{{}\Box}\nolimits}
\newcommand{\divergence}[1]{\inner{\vnabla}{#1}}
\newcommand{\ee}{\mathrm e}
\newcommand{\emesswert}{\del{\messwert \pm \messwert}}
\newcommand{\e}[1]{\cdot 10^{#1}}
\newcommand{\fehlt}{\textcolor{red}{Hier fehlen noch Inhalte.}}
\newcommand{\half}{\frac 12}
\newcommand{\ii}{\mathrm i}
\newcommand{\inner}[2]{\left\langle #1, #2 \right\rangle}
\newcommand{\laplace}{\mathop{{}\bigtriangleup}\nolimits}
\newcommand{\messwert}{\textcolor{blue}{\square}}
\newcommand{\punkte}{\textcolor{white}{xxxxx}}
\newcommand{\tens}[1]{\boldsymbol{#1}}
\newcommand{\vnabla}{\vec \nabla}
\renewcommand{\vec}[1]{\boldsymbol{#1}}

\newcommand{\themodul}{physik311}
\newcommand{\thegruppe}{Gruppe 3}
\newcommand{\theuebung}{2}

\pagestyle{fancy}

\fancyfoot[C]{\footnotesize{\thegruppe}}
\fancyfoot[L]{\footnotesize{Martin Ueding}}
\fancyfoot[R]{\footnotesize{Seite \thepage}}
\fancyhead[L]{\themodul{} -- Übung \theuebung}

\setcounter{section}{4}

\title{\themodul{} -- Übung \theuebung \\ \vspace{0.5cm} \large{\thegruppe}}

\author{Martin Ueding \\ \small{\href{mailto:mu@uni-bonn.de}{mu@uni-bonn.de}}}

\begin{document}

\maketitle

%%%%%%%%%%%%%%%%%%%%%%%%%%%%%%%%%%%%%%%%%%%%%%%%%%%%%%%%%%%%%%%%%%%%%%%%%%%%%%%
%                             Phasenverschiebung                              %
%%%%%%%%%%%%%%%%%%%%%%%%%%%%%%%%%%%%%%%%%%%%%%%%%%%%%%%%%%%%%%%%%%%%%%%%%%%%%%%

\section{Phasenverschiebung}

\subsection{Form}

Die Lichtgeschwindigkeit ist innerhalb der Glasplatte um $n$ kleiner, als in
der Umgebung. Die Welle hat im Vakuum eine Wellenzahl von:
\[
	k_1 = \frac \omega c
\]

Bei einer Glasdicke von $\Delta x$ würde die Welle bei voller Geschwindigkeit
(also $n=1$) um $\phi_1$ oszillieren:
\[
	\phi_1 = k_1 \Delta x = \frac \omega c \Delta x
\]

In der Glasplatte verändert sich allerdings die Wellenzahl zu:
\[
	k_n = \frac{\omega n}{c}
\]

In der Glasdicke können nun also mehr Schwingungen im gleichen Raum statt
finden, weil die räumliche Frequenz $k_n$ größer als $k_1$ ist:
\[
	\phi_n = k_n \Delta x = \frac{\omega n}c \Delta x
\]

Somit ist die Welle also um $\phi_n - \phi_1$:
\[
	\Delta \phi = \frac\omega c \del{n - 1} \Delta x
\]

Diese Phasenverschiebung lässt sich in der komplexen Schreibweise darstellen
als Vorfaktor:
\[
	\exp\del{-\ii \phi} = \exp\del{- \ii \frac\omega c \del{n - 1} \Delta x}
\]

Somit ist die Welle:
\[
	E = E_0 \exp\del{-\ii \del{\omega t - kx - k \del{n - 1} \Delta x}}
\]

Die Polarisation ist also um $\Delta \phi$ weitergedreht worden, im Vergleich
zur unbeeinflussten Ausbreitung.

\subsection{Vereinfachungen}

Für den Fall $n \approx 1$ und $\Delta x \ll 1$ wird die Welle wieder zur
vorherigen Welle, kann die Exponentialfunktion entwickelt werden:
\[
	E = E_0 \exp\del{-\ii \del{\omega t - kx}}\del{1+- \ii \frac\omega c \del{n - 1} \Delta x}
\]

Für $\del{n-1}\Delta x \approx 0$ kommt wieder die vorherige Welle heraus.

%%%%%%%%%%%%%%%%%%%%%%%%%%%%%%%%%%%%%%%%%%%%%%%%%%%%%%%%%%%%%%%%%%%%%%%%%%%%%%%
%                                Fata Morgana                                 %
%%%%%%%%%%%%%%%%%%%%%%%%%%%%%%%%%%%%%%%%%%%%%%%%%%%%%%%%%%%%%%%%%%%%%%%%%%%%%%%

\section{Fata Morgana}

\begin{figure}
	\centering
	\begin{tikzpicture}
		\draw (-2, -1) -- ++(4, 0);
		\draw (-2, 0) -- ++(4, 0);
		\draw[thick, ->] (0, 0) -- ++(30:2);
		\draw[thick, <-] (0, 0) -- ++(150:2);
		\draw (-1, 0) arc (180:150:1) node[below left] {$\phi$};
	\end{tikzpicture}
	\caption{Skizze zur Fata Morgana}
\end{figure}

\subsection{kritischer Winkel}

Gegeben ist, dass $\del{n-1} \propto \rho$ ist. Aus der idealen Gasgleichung
kann man $\rho \propto T^{-1}$ ableiten. Somit folgt:
\[
	\del{n-1} \propto \frac 1T
\]

Der Brechungsindex der unteren Luftschicht ist demnach:
\[
	n_2 = \del{n-1} \frac{T}{\Delta T + T} + 1
\]

Wenn der durchgehende Strahl im Winkel $\unit{\pi / 2}\rad$ gebrochen wird, ist
der kritische Winkel erreicht. Somit gilt nach dem Brechungsgesetz, in das
$\sin\del{\pi/2} = 1$ eingesetzt wurde:
\begin{align*}
	\sin\del{\frac \pi 2 - \phi} &= \frac{n_2}{n} \\
	\cos\del{\phi} &= \frac{n_2}{n} \\
	1-\frac{\phi^2}2 &= \frac{\del{n-1}\frac{T}{\Delta T + T}+1}{n} \\
	\intertext{Da $\del{n-1} \ll 1$ gelten soll, kann ich den Nenner entfallen lassen.}
	1-\frac{\phi^2}2 &= \del{n-1}\frac{T}{\Delta T + T}+1 \\
	\phi^2 &= 2\del{1-n}\frac{T}{\Delta T + T} \\
	  \phi &= \sqrt{2\del{1-n}\frac{T}{\Delta T + T}}
\end{align*}

Wie es jetzt allerdings weiter geht, weiß ich nicht.

\subsection{Abstand}

Bei $\phi = \unit{0.00447214}\rad$ und einer Höhe von $h := \unit{1.8}\meter$
ist der Abstand $d = \cot(\phi) h = \unit{402.49}\meter$.

%%%%%%%%%%%%%%%%%%%%%%%%%%%%%%%%%%%%%%%%%%%%%%%%%%%%%%%%%%%%%%%%%%%%%%%%%%%%%%%
%                     Licht als elektromagnetische Welle                      %
%%%%%%%%%%%%%%%%%%%%%%%%%%%%%%%%%%%%%%%%%%%%%%%%%%%%%%%%%%%%%%%%%%%%%%%%%%%%%%%

\section{Licht als elektromagnetische Welle}

\subsection{Feldstärken}

Die Lichtintensität $I$ ist der Betrag des Poyntingvektors $\vec S$:
%
\begin{align*}
	\vec S &= \vec E \times \vec H \\
	\intertext{
		Die gegebene Intensität ist wahrscheinlich ein Effektivwert, so dass
		ich hier ebenfalls mit den Effektivwerten rechne.
	}
	I &= E H \\
	I &= E \frac1{\mu_0} B \\
	\intertext{
		Dabei gilt für $E$ und $H$ in der elektromagnetischen Welle $E = c B$.
	}
	I &= \frac{1}{c\mu_0} E^2 \\
	\sqrt{I\mu_0 c} &= E = \unit{61.3784}{\volt\per\meter} \\
	B &= \unit{2.04736 \e{-7}}\tesla
\end{align*}

\subsection{Frequenz}

Die Frequenz von Licht ist:
\[
	c = f \lambda
	, \quad
	f = \unit{5.99 \e{14}}{\hertz}
\]

Eine Periode dauert $T = \frac 1f = \unit{1.67 \e{-15}}\second$. Direkt auf
einem Oszilloskop lässt sich dies nicht beobachten, dort ist bei einigen $\giga
\hertz$ Schluss. Vergleichen könnte man eine derartige Frequenz mit dem
Frequenzkamm. Im Vergleich zu alltäglichen Vorgängen ist eine Oszillation
absurd kurz. Es gibt allerdings gepulste Laser, die im $\atto \second$ Bereich
agieren, dagegen oszilliert das Licht langsam. Gegen die Plankzeit ist die
Periode gigantisch.

%%%%%%%%%%%%%%%%%%%%%%%%%%%%%%%%%%%%%%%%%%%%%%%%%%%%%%%%%%%%%%%%%%%%%%%%%%%%%%%
%                     Strahlenablenkung durch ein Prisma                      %
%%%%%%%%%%%%%%%%%%%%%%%%%%%%%%%%%%%%%%%%%%%%%%%%%%%%%%%%%%%%%%%%%%%%%%%%%%%%%%%

\section{Strahlenablenkung durch ein Prisma}

\subsection{Ablenkwinkel $\delta$}

Die Brechungswinkel im Inneren des Prismas seien $\theta_1'$ und $\theta_2'$.
Folgende Beziehungen lassen sich aus der Geometrie ableiten:
\[
	\theta_1 + \theta_2 + \pi - \delta + \pi - \alpha = 2 \pi
	,\quad
	\theta_1' + \theta_2' = \alpha
	,\quad
	\frac{\sin\del{\theta_1'}}{\sin\del{\theta_1}} = n = \frac{\sin\del{\theta_2'}}{\sin\del{\theta_2}}
\]

Daraus lässt sich \emph{irgendwie} die gesuchte Beziehung herleiten:
\[
	\delta = \theta_1 - \alpha + \arcsin\del{\sin\del\alpha \sqrt{n^2 - \sin^2\del{\theta_1}} - \cos\del\alpha \sin\del{\theta_1}}
\]

\subsection{minimales $\delta$}

Beim symmetrischen Durchgang durch das Prisma wird das Licht am wenigsten
gebrochen. Kommt es steiler auf der einen Seite rein, wird es dafür auf der
anderen Seite stärker gebrochen und umgekehrt. Dies ist besonders im Grenzfall
$\alpha \to 0$ zu beobachten.

Der minmale Ablenkwinkel ist gegeben durch:
\[
	\delta_\text{min} = 2 \arcsin\del{n \sin\del{\frac \alpha2}}
\]

Für das minimale $\delta$ gilt dann:
\[
	\theta_1 = \theta_2 = \half \del{\delta_\text{min} + \alpha}
\]

\subsection{konkrete Werte}

\begin{table}[h!]
	\centering
	\begin{tabular}{l|ccc}
		$n$ & $1.5130$ & $1.5067$ & $1.5246$ \\
		\hline
		$\delta_\text{min}$ & $\unit{1.71589}\rad$ & $\unit{1.70628}\rad$ & $\unit{1.73372}\rad$ \\
			  $\theta_1$ & $\unit{1.38154}\rad$ & $\unit{1.37674}\rad$ & $\unit{1.39046}\rad$ \\
	\end{tabular}
	\caption{verschiedene Werte für $n$ eingesetzt}
\end{table}

%\bibliography{../../zentrale_BibTeX/Central}
%\bibliographystyle{plain}

\end{document}

% vim: spell spelllang=de
