% Copyright © 2012-2013 Martin Ueding <dev@martin-ueding.de>
%
% Copyright © 2012 Martin Ueding <dev@martin-ueding.de>
%
\documentclass[11pt, ngerman, fleqn]{scrartcl}

\usepackage{graphicx}

%%%%%%%%%%%%%%%%%%%%%%%%%%%%%%%%%%%%%%%%%%%%%%%%%%%%%%%%%%%%%%%%%%%%%%%%%%%%%%%
%                                Locale, date                                 %
%%%%%%%%%%%%%%%%%%%%%%%%%%%%%%%%%%%%%%%%%%%%%%%%%%%%%%%%%%%%%%%%%%%%%%%%%%%%%%%

\usepackage{babel}
\usepackage[iso]{isodate}

%%%%%%%%%%%%%%%%%%%%%%%%%%%%%%%%%%%%%%%%%%%%%%%%%%%%%%%%%%%%%%%%%%%%%%%%%%%%%%%
%                          Margins and other spacing                          %
%%%%%%%%%%%%%%%%%%%%%%%%%%%%%%%%%%%%%%%%%%%%%%%%%%%%%%%%%%%%%%%%%%%%%%%%%%%%%%%

\usepackage[activate]{pdfcprot}
\usepackage[left=3cm, right=2cm, top=2cm, bottom=2cm]{geometry}
\usepackage[parfill]{parskip}
\usepackage{setspace}

\setlength{\columnsep}{2cm}

%%%%%%%%%%%%%%%%%%%%%%%%%%%%%%%%%%%%%%%%%%%%%%%%%%%%%%%%%%%%%%%%%%%%%%%%%%%%%%%
%                                    Color                                    %
%%%%%%%%%%%%%%%%%%%%%%%%%%%%%%%%%%%%%%%%%%%%%%%%%%%%%%%%%%%%%%%%%%%%%%%%%%%%%%%

\usepackage{color}

\definecolor{darkblue}{rgb}{0,0,.5}
\definecolor{darkgreen}{rgb}{0,.5,0}
\definecolor{darkred}{rgb}{.7,0,0}

%%%%%%%%%%%%%%%%%%%%%%%%%%%%%%%%%%%%%%%%%%%%%%%%%%%%%%%%%%%%%%%%%%%%%%%%%%%%%%%
%                         Font and font like settings                         %
%%%%%%%%%%%%%%%%%%%%%%%%%%%%%%%%%%%%%%%%%%%%%%%%%%%%%%%%%%%%%%%%%%%%%%%%%%%%%%%

\usepackage[charter, greekuppercase=italicized]{mathdesign}
\usepackage{beramono}
\usepackage{berasans}

% Style of vectors and tensors.
\newcommand{\tens}[1]{\boldsymbol{\mathsf{#1}}}
\renewcommand{\vec}[1]{\boldsymbol{#1}}

%%%%%%%%%%%%%%%%%%%%%%%%%%%%%%%%%%%%%%%%%%%%%%%%%%%%%%%%%%%%%%%%%%%%%%%%%%%%%%%
%                               Input encoding                                %
%%%%%%%%%%%%%%%%%%%%%%%%%%%%%%%%%%%%%%%%%%%%%%%%%%%%%%%%%%%%%%%%%%%%%%%%%%%%%%%

\usepackage[T1]{fontenc}
\usepackage[utf8]{inputenc}

%%%%%%%%%%%%%%%%%%%%%%%%%%%%%%%%%%%%%%%%%%%%%%%%%%%%%%%%%%%%%%%%%%%%%%%%%%%%%%%
%                         Hyperrefs and PDF metadata                          %
%%%%%%%%%%%%%%%%%%%%%%%%%%%%%%%%%%%%%%%%%%%%%%%%%%%%%%%%%%%%%%%%%%%%%%%%%%%%%%%

\usepackage{hyperref}
\usepackage{lastpage}

\hypersetup{
	breaklinks=false,
	citecolor=darkgreen,
	colorlinks=true,
	linkcolor=black,
	menucolor=black,
	pdfauthor={Martin Ueding},
	urlcolor=darkblue,
}

%%%%%%%%%%%%%%%%%%%%%%%%%%%%%%%%%%%%%%%%%%%%%%%%%%%%%%%%%%%%%%%%%%%%%%%%%%%%%%%
%                               Math Operators                                %
%%%%%%%%%%%%%%%%%%%%%%%%%%%%%%%%%%%%%%%%%%%%%%%%%%%%%%%%%%%%%%%%%%%%%%%%%%%%%%%

\usepackage[thinspace, squaren]{SIunits}
\usepackage{amsmath}
\usepackage{amsthm}
\usepackage{commath}

% Word like operators.
\DeclareMathOperator{\arcsinh}{arsinh}
\DeclareMathOperator{\arsinh}{arsinh}
\DeclareMathOperator{\asinh}{arsinh}
\DeclareMathOperator{\card}{card}
\DeclareMathOperator{\diam}{diam}
\renewcommand{\Im}{\mathop{{}\mathrm{Im}}\nolimits}
\renewcommand{\Re}{\mathop{{}\mathrm{Re}}\nolimits}

% Special single letters.
\DeclareMathOperator{\fourier}{\mathcal{F}}
\newcommand{\C}{\ensuremath{\mathbb C}}
\newcommand{\ee}{\mathrm e}
\newcommand{\ii}{\mathrm i}
\newcommand{\N}{\ensuremath{\mathbb N}}
\newcommand{\R}{\ensuremath{\mathbb R}}

% Shape like operators.
\DeclareMathOperator{\dalambert}{\Box}
\DeclareMathOperator{\laplace}{\bigtriangleup}
\newcommand{\curl}{\vnabla \times}
\newcommand{\divergence}[1]{\inner{\vnabla}{#1}}
\newcommand{\vnabla}{\vec \nabla}

% Shortcuts
\newcommand{\ev}{\hat{\vec e}}
\newcommand{\e}[1]{\cdot 10^{#1}}
\newcommand{\half}{\frac 12}
\newcommand{\inner}[2]{\left\langle #1, #2 \right\rangle}

% Placeholders.
\newcommand{\emesswert}{\del{\messwert \pm \messwert}}
\newcommand{\fehlt}{\textcolor{darkred}{Hier fehlen noch Inhalte.}}
\newcommand{\messwert}{\textcolor{blue}{\square}}
\newcommand{\punkte}{\textcolor{white}{xxxxx}}


\usepackage{float}
\usepackage{scrpage2}
\usepackage{tikz}

\newcommand{\themodul}{physik311}
\newcommand{\thegruppe}{Gruppe 3 -- Matthias Rehberger}
\newcommand{\theuebung}{13}

\pagestyle{scrheadings}

\cfoot{\footnotesize{\thegruppe}}
\ifoot{\footnotesize{Martin Ueding}}
\ofoot{\footnotesize{Seite \thepage\ / \pageref{LastPage}}}
\ihead{\themodul{} -- Übung \theuebung}
\ohead{\rightmark}
\chead{}
\setheadsepline{.4pt}
\automark{section}

\setcounter{section}{44}


\def\thesubsection{\thesection\alph{subsection}}

\title{\themodul{} -- Übung \theuebung \\ \vspace{0.5cm} \large{\thegruppe}}

\author{Martin Ueding \\ \small{\href{mailto:mu@uni-bonn.de}{mu@uni-bonn.de}}}

\begin{document}

\maketitle

\begin{table}[h]
	\centering
	\begin{tabular}{l|c|c|c|c}
		Aufgabe
		& \ref 1
		& \ref 2
		& \ref 3
		& $\sum$ \\
		\hline
		Punkte
		& \punkte
		& \punkte
		& \punkte
		& \punkte
	\end{tabular}
\end{table}

%%%%%%%%%%%%%%%%%%%%%%%%%%%%%%%%%%%%%%%%%%%%%%%%%%%%%%%%%%%%%%%%%%%%%%%%%%%%%%%
%                          de Broglie und Buckyballs                          %
%%%%%%%%%%%%%%%%%%%%%%%%%%%%%%%%%%%%%%%%%%%%%%%%%%%%%%%%%%%%%%%%%%%%%%%%%%%%%%%

\section{de Broglie und Buckyballs}
\label 1

\subsection{Wellenlänge}

Es ist eine Beugung am Gitter, es gilt mit dem Spaltabstand $d$, der
Schirmentfernung $l$, dem Abstand des $n$-ten Maximum von der Mitte des Schirms
$x$:
\[
	\frac xl d = n \lambda
\]

Für $n = 1$ lese ich im Plot $x = \unit{45}{\micro\meter}$ ab. Mit $l =
\unit{1.25}\meter$ und $d = \unit{100}{\nano\meter}$ errechne ich $\lambda$:
\[
	\lambda = \unit{3.6}{\pico\meter}
\]

Diese ist deutlich kleiner als deren Durchmesser. Dies liegt an der hohen
Geschwindigkeit.

\subsection{Austrittsgeschwindigkeit}

Mit $p = h/\lambda$ errechne ich den Impuls:
\[
	p = \unit{1.84 \e{-22}}{\kilogram\usk\meter\per\second}
\]

Die Masse der Teilchen sind jeweils $m = \unit{60 \cdot 12}\atomicmass =
\unit{1.20 \e{-24}}\kilogram$.

Die (nichtrelativistische) Geschwindigkeit ist:
\[
	v = \frac pm = \unit{154}{\meter\per\second}
\]

Mit einer thermischen Energie $E = kT/2$ kann ich die Temperatur ausdrücken:
\[
	T = \frac{p^2}{km} = \unit{2050}\kelvin
\]

\subsection{Rubidium}

Mit
\[
	\frac{p^2}{2m} = \half kT
\]
kann ich nach $\lambda$ umstellen:
\[
	\lambda = \frac{h}{\sqrt{kTm}} = \unit{1.48}{\micro\meter}
\]

%%%%%%%%%%%%%%%%%%%%%%%%%%%%%%%%%%%%%%%%%%%%%%%%%%%%%%%%%%%%%%%%%%%%%%%%%%%%%%%
%                            Lichtdruck auf Atome                             %
%%%%%%%%%%%%%%%%%%%%%%%%%%%%%%%%%%%%%%%%%%%%%%%%%%%%%%%%%%%%%%%%%%%%%%%%%%%%%%%

\section{Lichtdruck auf Atome}
\label 2

\paragraph{Atome}

Der Impulsübertrag pro Photon ist:
\[
	p = 2 \frac{hf}{c}
	= 2 \frac h\lambda
	= \unit{1.70\e{-27}}{\kilogram\usk\meter\per\second}
\]

Dies passiert jede $\tau/2 = \unit{14}{\nano\second}$, da der Hin- und
Rückübergang vollzogen werden muss. Somit ist die Kraft:
\[
	F = \frac p\tau = \unit{31}{\zepto\newton}
\]

Bei einer Masse von $m = \unit{87}\atomicmass$ ist dies eine Beschleunigung
von:
\[
	a = \frac Fm = \unit{214}{\kilo\meter\per\second\squared}
\]

\paragraph{Erdbeschleunigung}

Die Beschleunigung durch die Erde ist $\unit{9.81}{\meter\per\second\squared}$,
was einer Kraft von $F_G = \unit{1.42}{\yocto\newton}$ entspricht.

\paragraph{Golfball}

Der Lichtdruck $P$ auf eine Fläche $A$ bei einer Leistung $\dot W$:
\[
	P = \frac FA = \frac{\dot W}{Ac}
\]

Ich löse nach $F$ auf und erhalte:
\[
	F = \unit{16.7}{\nano\newton}
\]

Dies ist eine Beschleunigung von $a_B = \unit{37}{\micro\meter\per\second\squared}$.

Alles zusammen zur Übersicht:

\begin{tabular}{lrr}
	Objekt & Kraft $F$ & Beschleunigung $a$ \\
	\hline
	Atom -- Laser
	& $\unit{31}{\zepto\newton}$
	& $\unit{214}{\kilo\meter\per\second\squared}$ \\
	Atom -- Schwerkraft
	& $\unit{1.42}{\yocto\newton}$
	& $\unit{9.81}{\meter\per\second\squared}$ \\
	Golfball -- Laser
	& $\unit{16.7}{\nano\newton}$
	& $\unit{37}{\micro\meter\per\second\squared}$
\end{tabular}

%%%%%%%%%%%%%%%%%%%%%%%%%%%%%%%%%%%%%%%%%%%%%%%%%%%%%%%%%%%%%%%%%%%%%%%%%%%%%%%
%                      Schwarzkörperstrahlung der Sonne                      %
%%%%%%%%%%%%%%%%%%%%%%%%%%%%%%%%%%%%%%%%%%%%%%%%%%%%%%%%%%%%%%%%%%%%%%%%%%%%%%%

\section{Schwarzkörperstrahlung der Sonne}
\label 3

Mit dem Fluss $F$, der Entfernung Erde--Sonne $R$ kann ich die Leuchtleistung
$P$ der Sonne bestimmen:
\[
	P = 4 \pi R^2 F = \unit{395}{\yotta\watt}
\]

Die Temperatur $T$ errechne ich über das Stefan-Boltzmann-Gesetz mit dem
Sonnenradius $r$:
\[
	T = \sqrt[4]{\frac{P}{4 \pi r^2 \sigma}} = \unit{5800}{\kelvin}
\]

Die maximale Frequenz erhalte ich über das Wien'sche Verschiebungsgesetz:
\[
	\nu = \unit{58.8}{\nano\hertz\per\kelvin} \cdot T = \unit{340}{\tera\hertz}
\]

Dies entspricht einer Wellenlänge von:
\[
	\lambda = \frac c\nu = \unit{879}{\nano\meter}
\]

Allerdings ist die die Wellenlänge zur Frequenz mit der maximalen Leistung $\od P\nu$, nicht die Wellenlänge zur maximalen Leistung $\od P\lambda$. Letzere berechnet sich mit:
\[
	\lambda' = \frac{\unit{2.89}{\milli\meter\per\kelvin}}T
	= \unit{500}{\nano\meter}
\]

Dies ist grün, passt also.

%\bibliography{../../zentrale_BibTeX/Central}
%\bibliographystyle{plain}

\end{document}

% vim: spell spelllang=de
