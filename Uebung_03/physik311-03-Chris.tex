\documentclass[11pt]{article}
\usepackage[T1]{fontenc}
\usepackage[a4paper, left=3cm, right=2cm, top=2cm, bottom=2cm]{geometry}
\usepackage[activate]{pdfcprot}
\usepackage[ngerman]{babel}
\usepackage[parfill]{parskip}
\usepackage[math]{iwona}
\usepackage[utf8]{inputenc}
\usepackage{amsmath}
\usepackage{amssymb}
\usepackage{color}
\usepackage{fancyhdr}
\usepackage{graphicx}
\usepackage{hyperref}
\usepackage[cdot, squaren]{SIunits}
\usepackage{setspace}
\usepackage{units}


\definecolor{darkblue}{rgb}{0,0,.5}
\hypersetup{pdftex=true, colorlinks=true, breaklinks=false, linkcolor=black, menucolor=black, pagecolor=black, urlcolor=darkblue}

\setlength{\columnsep}{2cm}

\newcommand{\arcsinh}{\mathrm{arcsinh}}
\newcommand{\asinh}{\mathrm{arcsinh}}
\newcommand{\ergebnis}{\textcolor{red}{\mathrm{Ergebnis}}}
\newcommand{\fehlt}{\textcolor{red}{Hier fehlen noch Inhalte.}}
\newcommand{\half}{\frac{1}{2}}
\renewcommand{\d}{\, \mathrm d}
\newcommand{\punkte}{\textcolor{white}{xxxxx}}
\newcommand{\p}{\, \partial}

\newcommand{\themodul}{physik311}
\newcommand{\thetutor}{Jens Ulitzsch}
\newcommand{\theuebung}{3}

\pagestyle{fancy}
\fancyhead[L]{\themodul{} -- Übung \theuebung}
\fancyfoot[C]{\footnotesize{Tutor: \thetutor}}
\fancyfoot[L]{\footnotesize{Christoph Hansen}}
\fancyfoot[R]{\footnotesize{Seite \thepage}}


\title{\themodul{} Übung \theuebung \\ Tutor: \thetutor}

\author{Christoph Hansen \\ {\small \href{mailto:christophhansen@uni-bonn.de}{christophhansen@uni-bonn.de}}}

\begin{document}

\maketitle


\section*{Aufgabe 9 Katzenauge}

\includegraphics[scale=1]{Aufgabe_9.png}

Man sieht, das der Winkel $\alpha$ ein Wechselwinkel ist und damit rechts neben dem Winkel $\beta$ nochmal auftaucht. Damit ergibt sich $\alpha + \beta = 90^{\circ}$.
Der Winkel $\beta$ springt genauso neben das $\alpha$ und es ergeben sich wieder $90^{\circ}$. Man hat also zweimal $90^{\circ}$ und damit $180^{\circ}$, die sich unabhängig vom Einfallswinkel einstellen.

\section*{Aufgabe 10 Phasen- und Gruppengeschwindigkeit}

\subsection*{}

Da ich hier nur eine Phase der Welle betrachte ist der Term $\left(k(\omega)\cdot z - \omega t \right)= const$, deshalb forme ich nach $z$ um:
\begin{align*}
z &=  \frac{\omega t}{k(\omega)} + const \\
\intertext{Da ich einen Ausdruck für die Geschwindigkeit haben möchte leite ich $z$ nach der Zeit ab:} \\
v_{ph} &= \frac{\omega}{k(\omega)}
\intertext{Mit der Definition von $k(\omega)$ ergibt sich die gesuchte Relation} \\
v_{ph} &= \frac{\omega}{\frac{\omega \cdot n(\omega)}{c}} = \frac{c}{n(\omega)}
\end{align*}

\subsection*{}

Ich verstehe die Gruppengeschwindigkeit als Gesamtgeschwindigkeit der einzelnen Phasen in der Welle. Ich nehme mir also zwei Wellen mit den Kreisfrequenzen $\omega$ und $\omega + \Delta \omega$, sowie der Kreiswellenzahl $k$ und $k + \Delta k$. \\
Das kann ich jetzt darauf untersuchen, wann die beiden Wellen übereinander liegen:
\begin{align*}
kx - \omega t &= \left( k + \Delta k \right)x - \left( \omega + \Delta \omega \right)t \\
\intertext{einfacher:} \\
\Delta kx &= \Delta \omega t \\
\intertext{Bei $t = 0$ stimmen die beiden Wellen überein und etwas später bei $t = \Delta t$ an der Stelle $\Delta x = \Delta t \Delta \omega / \Delta k$.
Die Wellengruppe hat sich also mit der Geschwindigkeit} \\
v_G &= \frac{d \omega}{d k}
\intertext{verschoben.}
\end{align*}

\section*{Aufgabe 11 Phasen- und Gruppengeschwindigkeit in Medien}

Die Gleichung für die Gruppen- und Phasengeschwindigkeit haben wir ja schon in der vorigen Aufgabe berechnet, wir können die neue Relation für $n(\omega)$ also einfach einsetzten:
\begin{align*}
v_{ph} &= \frac{c}{1 + \frac{Ne^2}{2\epsilon_0 m \left( \omega_0^2 - \omega^2 + i \omega \gamma \right)}} \\
v_g &= \frac{1}{\frac{d k}{d \omega}} = \frac{1}{ \frac{d k}{d \omega} \left( \frac{\omega}{c} \left( 1+\frac{Ne^2}{2\epsilon_0m (\omega_0^2 - \omega + i\omega \gamma) \right) \right)}} \\
&= \frac{c}{1+\frac{Ne^2(2c\epsilon_0m + 2c\epsilon_0m \omega^2)}{\left(2c\epsilon_0m + \omega \left( i \gamma 2 \epsilon_0m - 2c\epsilon_0 m \omega \right) \right)^2}}
\end{align*} 



\section*{Aufgabe 12 Fermatsches Prinzip und Reflexion}

\includegraphics[scale=1]{Aufgabe_12.png}

Für die Kennzeichnung der Strecken bitte die Skizze beachten.
Der von Punkt $P_1$ kommende Strahl wird im Punkt $A$ reflektiert und läuft dann durch den Punkt $P_2$. \\
Ich betrachte nun den Weg, den das Licht zurücklegen muss um von Punkt $P_1$ nach $P_2$ zu gelangen. Es ergibt sich dabei:
\begin{align*}
s &= \sqrt{a^2+1} + \sqrt{(a-1)^2+1} \\
\intertext{Davon betrachte ich die zeitliche Ableitung, da ich wissen möchte wo das Minimum liegt} \\
\frac{ds}{da} &= \frac{2a}{2\sqrt{a^2+1}} - \frac{2(a-1)}{2\sqrt{(a-1)^2+1}} \overset{!}{=} 0 \\
\intertext{Es gilt also:} \\
\frac{a}{\sqrt{a^2+1}} &= \frac{a-1}{\sqrt{(a-1)^2+1}} \\
\intertext{Ich stelle für den Sinus folgende Beziehungen auf:} \\
\sin(\alpha) &= \frac{a}{\sqrt{a^2+1}} \\
\sin(\alpha') &= \frac{a-1}{\sqrt{(a-1)^2+1}} 
\intertext{Diese setzte ich nun ein} \\
\sin(\alpha) &= \sin(\alpha') \\
\alpha &= \alpha'
\end{align*}
Damit wäre gezeigt, das der Einfallswinkel gleich dem Ausfallswinkel ist.


\end{document}

