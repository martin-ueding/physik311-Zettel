% Copyright © 2012 Martin Ueding <dev@martin-ueding.de>
%
\documentclass[11pt, ngerman, fleqn]{article}

\usepackage[a4paper, left=3cm, right=2cm, top=2cm, bottom=2cm]{geometry}
\usepackage[activate]{pdfcprot}
\usepackage[cdot, squaren]{SIunits}
\usepackage[iso]{isodate}
\usepackage[parfill]{parskip}
\usepackage[T1]{fontenc}
\usepackage[utf8]{inputenc}
\usepackage{amsmath}
\usepackage{amsthm}
\usepackage{babel}
\usepackage{color}
\usepackage{lastpage}
\usepackage{commath}
\usepackage{epstopdf}
\usepackage{fancyhdr}
\usepackage{graphicx}
\usepackage{hyperref}
\usepackage{setspace}
\usepackage{tikz}

\usepackage[charter, greekuppercase=italicized]{mathdesign}

\usetikzlibrary{intersections}

\definecolor{darkblue}{rgb}{0,0,.5}
\definecolor{darkgreen}{rgb}{0,.5,0}

\hypersetup{
	breaklinks=false,
	citecolor=darkgreen,
	colorlinks=true,
	linkcolor=black,
	menucolor=black,
	urlcolor=darkblue,
}

\setlength{\columnsep}{1.5cm}

\DeclareMathOperator{\arcsinh}{arsinh}
\DeclareMathOperator{\arsinh}{arsinh}
\DeclareMathOperator{\asinh}{arsinh}
\DeclareMathOperator{\card}{card}
\DeclareMathOperator{\diam}{diam}

\newcommand{\dalambert}{\mathop{{}\Box}\nolimits}
\newcommand{\divergence}[1]{\inner{\vnabla}{#1}}
\newcommand{\ee}{\mathrm e}
\newcommand{\emesswert}{\del{\messwert \pm \messwert}}
\newcommand{\e}[1]{\cdot 10^{#1}}
\newcommand{\fehlt}{\textcolor{red}{Hier fehlen noch Inhalte.}}
\newcommand{\half}{\frac 12}
\newcommand{\ii}{\mathrm i}
\newcommand{\inner}[2]{\left\langle #1, #2 \right\rangle}
\newcommand{\laplace}{\mathop{{}\bigtriangleup}\nolimits}
\newcommand{\messwert}{\textcolor{blue}{\square}}
\newcommand{\punkte}{\textcolor{white}{xxxxx}}
\newcommand{\tens}[1]{\boldsymbol{#1}}
\newcommand{\vnabla}{\vec \nabla}
\renewcommand{\vec}[1]{\boldsymbol{#1}}

\newcommand{\themodul}{physik311}
\newcommand{\thegruppe}{Gruppe 3 -- Matthias Rehberger}
\newcommand{\theuebung}{3}

\pagestyle{fancy}

\fancyfoot[C]{\footnotesize{\thegruppe}}
\fancyfoot[L]{\footnotesize{Martin Ueding}}
\fancyfoot[R]{\footnotesize{Seite \thepage\ / \pageref{LastPage}}}
\fancyhead[L]{\themodul{} -- Übung \theuebung}

\setcounter{section}{8}

\def\thesubsection{\thesection\alph{subsection}}

\title{\themodul{} -- Übung \theuebung \\ \vspace{0.5cm} \large{\thegruppe}}

\author{Martin Ueding \\ \small{\href{mailto:mu@uni-bonn.de}{mu@uni-bonn.de}}}

\begin{document}

\maketitle

%%%%%%%%%%%%%%%%%%%%%%%%%%%%%%%%%%%%%%%%%%%%%%%%%%%%%%%%%%%%%%%%%%%%%%%%%%%%%%%
%                                  Reflektor                                  %
%%%%%%%%%%%%%%%%%%%%%%%%%%%%%%%%%%%%%%%%%%%%%%%%%%%%%%%%%%%%%%%%%%%%%%%%%%%%%%%

\section{Reflektor}

In Abbildung \ref{fig:ref} ist leicht zu sehen, dass der alle Winkel zusammen
sind:
\[
	\alpha + \alpha + \del{\half \pi - \alpha} + \del{\half \pi - \alpha} = \pi
\]

Somit ist es eine Reflektion um $\pi$.

\begin{figure}[h]
	\centering
	\includegraphics[width=.3\textheight]{Reflektor.pdf}
	\caption{Skizze zum Reflektor}
	\label{fig:ref}
\end{figure}

%%%%%%%%%%%%%%%%%%%%%%%%%%%%%%%%%%%%%%%%%%%%%%%%%%%%%%%%%%%%%%%%%%%%%%%%%%%%%%%
%                     Phasen- und Gruppengeschwindigkeit                      %
%%%%%%%%%%%%%%%%%%%%%%%%%%%%%%%%%%%%%%%%%%%%%%%%%%%%%%%%%%%%%%%%%%%%%%%%%%%%%%%

\section{Phasen- und Gruppengeschwindigkeit}

\subsection{monochromatische Welle}

Zuerst führe ich die Fouriertransformation aus. Dabei ist das durch die Deltadistribution recht einfach:
\[
	E(z, t)
	= \int_{-\infty}^\infty \dif \omega E_0 \delta\del{\omega - \omega_m} \exp\del{i\del{k(\omega) z - \omega t}}
	= E_0 \exp\del{i\del{k(\omega_m) z - \omega_m t}}
\]

Die Phase ist konstant, wenn das Argument der Exponentialfunktion konstant ist. Damit kann ich ableiten:
\[
	\dpd{}t \del{k\del{\omega_m} z - \omega t} = 0
	\quad \Leftrightarrow \quad
	\dot z = \frac{c}{n\del{\omega_m}}
\]

Dabei ist $\dot z$ die Phasengeschwindigkeit in $z$-Richtung.

%%%%%%%%%%%%%%%%%%%%%%%%%%%%%%%%%%%%%%%%%%%%%%%%%%%%%%%%%%%%%%%%%%%%%%%%%%%%%%%
%                Phasen- und Gruppengeschwindigkeit in Medien                 %
%%%%%%%%%%%%%%%%%%%%%%%%%%%%%%%%%%%%%%%%%%%%%%%%%%%%%%%%%%%%%%%%%%%%%%%%%%%%%%%

\section{Phasen- und Gruppengeschwindigkeit in Medien}

Diese Aufgabe lasse ich aus.

%%%%%%%%%%%%%%%%%%%%%%%%%%%%%%%%%%%%%%%%%%%%%%%%%%%%%%%%%%%%%%%%%%%%%%%%%%%%%%%
%                      fermatsches Prinzip und Relextion                      %
%%%%%%%%%%%%%%%%%%%%%%%%%%%%%%%%%%%%%%%%%%%%%%%%%%%%%%%%%%%%%%%%%%%%%%%%%%%%%%%

\section{fermatsches Prinzip und Relextion}

Hier kann ich einfach die Herleitung für das Brechungsgesetz wiederbenutzen, in dem ich $n_1 = n_2$ setze.

Es soll das Reflektionsgesetz hergeleitet werden:
\[ \frac{\sin\del{\theta_1}}{\sin\del{\theta_2}} = 1 \]

Angenommen, das Licht beginnt im Punkt $P_1(0, 1)$ und bewegt sich zum Punkt
$P_2(1, 1)$. Das Medium wird im Punkt $(A, 0)$ gewechselt. Somit ist die Strecke, die das Licht zurücklegen muss:
\[ \sqrt{1 + A^2} + \sqrt{(1-A)^2 + 1^2} \]

Interessant ist allerdings die Zeit, die das Licht braucht:
\[ t := \frac 1{c} \sqrt{1 + A^2} + \frac 1{c} \sqrt{(1-A)^2 + 1} \]

Ich kann $A$ und $(1-A)$ auch durch den Sinus ausdrücken:
\[
	\sin\del{\theta_1} = \frac{A}{\sqrt{1 + A^2}}
	, \quad
	\sin\del{\theta_2} = \frac{1 - A}{\sqrt{1 + (1-A)^2}}
\]

Da eine extremale Zeit gefunden werden soll, leite ich nach $A$ ab und setze
gleich $0$:
%
\begin{align*}
\dpd t A &= 0 \\
\frac{2A}{2 c \sqrt{1+A^2}} - \frac{2(1-A)}{2 c \sqrt{(1-A^2)+1}} &= 0 \\
\frac{A}{\sqrt{1+A^2}} &= \frac{(1-A)}{\sqrt{(1-A^2)+1}} \\
1 &= \frac{\frac{(1-A)}{\sqrt{(1-A^2)+1}}}{\frac{A}{\sqrt{1+A^2}}} \\
\intertext{Nun kann ich die Winkel einsetzen.}
1 &= \frac{\sin\del{\theta_2}}{\sin\del{\theta_1}}
\end{align*}

Es ist anschaulich klar, dass dieses Extremum ein Minimum ist. Somit ist die
letzte Zeile die gesuchte Relation.

Daraus folgt direkt:
\[ \theta_1 = \theta_2 \]

%\bibliography{../../zentrale_BibTeX/Central}
%\bibliographystyle{plain}

\end{document}

% vim: spell spelllang=de
