% Copyright © 2012 Martin Ueding <dev@martin-ueding.de>
%
% Copyright © 2012 Martin Ueding <dev@martin-ueding.de>
%
\documentclass[11pt, ngerman, fleqn]{scrartcl}

\usepackage{graphicx}

%%%%%%%%%%%%%%%%%%%%%%%%%%%%%%%%%%%%%%%%%%%%%%%%%%%%%%%%%%%%%%%%%%%%%%%%%%%%%%%
%                                Locale, date                                 %
%%%%%%%%%%%%%%%%%%%%%%%%%%%%%%%%%%%%%%%%%%%%%%%%%%%%%%%%%%%%%%%%%%%%%%%%%%%%%%%

\usepackage{babel}
\usepackage[iso]{isodate}

%%%%%%%%%%%%%%%%%%%%%%%%%%%%%%%%%%%%%%%%%%%%%%%%%%%%%%%%%%%%%%%%%%%%%%%%%%%%%%%
%                          Margins and other spacing                          %
%%%%%%%%%%%%%%%%%%%%%%%%%%%%%%%%%%%%%%%%%%%%%%%%%%%%%%%%%%%%%%%%%%%%%%%%%%%%%%%

\usepackage[activate]{pdfcprot}
\usepackage[left=3cm, right=2cm, top=2cm, bottom=2cm]{geometry}
\usepackage[parfill]{parskip}
\usepackage{setspace}

\setlength{\columnsep}{2cm}

%%%%%%%%%%%%%%%%%%%%%%%%%%%%%%%%%%%%%%%%%%%%%%%%%%%%%%%%%%%%%%%%%%%%%%%%%%%%%%%
%                                    Color                                    %
%%%%%%%%%%%%%%%%%%%%%%%%%%%%%%%%%%%%%%%%%%%%%%%%%%%%%%%%%%%%%%%%%%%%%%%%%%%%%%%

\usepackage{color}

\definecolor{darkblue}{rgb}{0,0,.5}
\definecolor{darkgreen}{rgb}{0,.5,0}
\definecolor{darkred}{rgb}{.7,0,0}

%%%%%%%%%%%%%%%%%%%%%%%%%%%%%%%%%%%%%%%%%%%%%%%%%%%%%%%%%%%%%%%%%%%%%%%%%%%%%%%
%                         Font and font like settings                         %
%%%%%%%%%%%%%%%%%%%%%%%%%%%%%%%%%%%%%%%%%%%%%%%%%%%%%%%%%%%%%%%%%%%%%%%%%%%%%%%

\usepackage[charter, greekuppercase=italicized]{mathdesign}
\usepackage{beramono}
\usepackage{berasans}

% Style of vectors and tensors.
\newcommand{\tens}[1]{\boldsymbol{\mathsf{#1}}}
\renewcommand{\vec}[1]{\boldsymbol{#1}}

%%%%%%%%%%%%%%%%%%%%%%%%%%%%%%%%%%%%%%%%%%%%%%%%%%%%%%%%%%%%%%%%%%%%%%%%%%%%%%%
%                               Input encoding                                %
%%%%%%%%%%%%%%%%%%%%%%%%%%%%%%%%%%%%%%%%%%%%%%%%%%%%%%%%%%%%%%%%%%%%%%%%%%%%%%%

\usepackage[T1]{fontenc}
\usepackage[utf8]{inputenc}

%%%%%%%%%%%%%%%%%%%%%%%%%%%%%%%%%%%%%%%%%%%%%%%%%%%%%%%%%%%%%%%%%%%%%%%%%%%%%%%
%                         Hyperrefs and PDF metadata                          %
%%%%%%%%%%%%%%%%%%%%%%%%%%%%%%%%%%%%%%%%%%%%%%%%%%%%%%%%%%%%%%%%%%%%%%%%%%%%%%%

\usepackage{hyperref}
\usepackage{lastpage}

\hypersetup{
	breaklinks=false,
	citecolor=darkgreen,
	colorlinks=true,
	linkcolor=black,
	menucolor=black,
	pdfauthor={Martin Ueding},
	urlcolor=darkblue,
}

%%%%%%%%%%%%%%%%%%%%%%%%%%%%%%%%%%%%%%%%%%%%%%%%%%%%%%%%%%%%%%%%%%%%%%%%%%%%%%%
%                               Math Operators                                %
%%%%%%%%%%%%%%%%%%%%%%%%%%%%%%%%%%%%%%%%%%%%%%%%%%%%%%%%%%%%%%%%%%%%%%%%%%%%%%%

\usepackage[thinspace, squaren]{SIunits}
\usepackage{amsmath}
\usepackage{amsthm}
\usepackage{commath}

% Word like operators.
\DeclareMathOperator{\arcsinh}{arsinh}
\DeclareMathOperator{\arsinh}{arsinh}
\DeclareMathOperator{\asinh}{arsinh}
\DeclareMathOperator{\card}{card}
\DeclareMathOperator{\diam}{diam}
\renewcommand{\Im}{\mathop{{}\mathrm{Im}}\nolimits}
\renewcommand{\Re}{\mathop{{}\mathrm{Re}}\nolimits}

% Special single letters.
\DeclareMathOperator{\fourier}{\mathcal{F}}
\newcommand{\C}{\ensuremath{\mathbb C}}
\newcommand{\ee}{\mathrm e}
\newcommand{\ii}{\mathrm i}
\newcommand{\N}{\ensuremath{\mathbb N}}
\newcommand{\R}{\ensuremath{\mathbb R}}

% Shape like operators.
\DeclareMathOperator{\dalambert}{\Box}
\DeclareMathOperator{\laplace}{\bigtriangleup}
\newcommand{\curl}{\vnabla \times}
\newcommand{\divergence}[1]{\inner{\vnabla}{#1}}
\newcommand{\vnabla}{\vec \nabla}

% Shortcuts
\newcommand{\ev}{\hat{\vec e}}
\newcommand{\e}[1]{\cdot 10^{#1}}
\newcommand{\half}{\frac 12}
\newcommand{\inner}[2]{\left\langle #1, #2 \right\rangle}

% Placeholders.
\newcommand{\emesswert}{\del{\messwert \pm \messwert}}
\newcommand{\fehlt}{\textcolor{darkred}{Hier fehlen noch Inhalte.}}
\newcommand{\messwert}{\textcolor{blue}{\square}}
\newcommand{\punkte}{\textcolor{white}{xxxxx}}


\usepackage{scrpage2}
\usepackage{tikz}

\newcommand{\themodul}{physik311}
\newcommand{\thegruppe}{Gruppe 3 -- Matthias Rehberger}
\newcommand{\theuebung}{10}

\pagestyle{scrheadings}

\cfoot{\footnotesize{\thegruppe}}
\ifoot{\footnotesize{Martin Ueding}}
\ofoot{\footnotesize{Seite \thepage\ / \pageref{LastPage}}}
\ihead{\themodul{} -- Übung \theuebung}
\ohead{\rightmark}
\chead{}
\setheadsepline{.4pt}
\automark{section}

\setcounter{section}{32}


\def\thesubsection{\thesection\alph{subsection}}

\title{\themodul{} -- Übung \theuebung \\ \vspace{0.5cm} \large{\thegruppe}}

\author{Martin Ueding \\ \small{\href{mailto:mu@uni-bonn.de}{mu@uni-bonn.de}}}

\begin{document}

\maketitle

\begin{table}[h]
	\centering
	\begin{tabular}{l|c|c|c|c|c}
		Aufgabe & \ref 1 & \ref 2 & \ref 3 & \ref 4 & $\sum$   \\
		\hline
		Punkte & \punkte & \punkte & \punkte & \punkte & \punkte
	\end{tabular}
\end{table}

%%%%%%%%%%%%%%%%%%%%%%%%%%%%%%%%%%%%%%%%%%%%%%%%%%%%%%%%%%%%%%%%%%%%%%%%%%%%%%%
%                       dielektrische Vielfachschichten                       %
%%%%%%%%%%%%%%%%%%%%%%%%%%%%%%%%%%%%%%%%%%%%%%%%%%%%%%%%%%%%%%%%%%%%%%%%%%%%%%%

\section{dielektrische Vielfachschichten}
\label 1

\subsection{Einsetzen und Ausrechnen für Dreifachschicht}

Diese Aufgabe habe ich in \emph{Mathematica} gerechnet. \footnote{Die Datei ist
auf \url{http://uni-bonn.de/~s6mauedi/physik311/} zu finden.} Dazu habe ich
zuerst einige Variablen definiert:
\begin{gather*}
	n_0 := 1
	,\quad
	n_1 := 1.38
	,\quad
	n_2 := 2.30
	,\quad
	n_3 := 1.76
	,\quad
	n_4 := 1.63 \\
	%
	\delta_j := \frac{2 \pi}{\lambda_0} n_j 2 L_j \cos\del{\theta_j} \\
	%
	L_1 := \frac{\lambda_0}{4 n_1}
	,\quad
	L_2 := \frac{2 \lambda_0}{4 n_3}
	,\quad
	L_3 := \frac{3 \lambda_0}{4 n_3} \\
	%
	\theta_j := 0 \\
	% \\
	\tens M_j := \begin{pmatrix}
		\cos\del{\frac{\delta_j}{2}} &
		\ii \sin\del{\frac{\delta_j}{2}} \frac{1}{n_j} \\
		\ii \sin\del{\frac{\delta_j}{2}} n_j &
		\cos\del{\frac{\delta_j}{2}}
	\end{pmatrix} \\
	%
	\tens P := \tens M_3 \tens M_2 \tens M_1 \\
	%
	\tens E :=
	\begin{pmatrix}
		n_0 \\ -1
	\end{pmatrix}
	\tens P
	\begin{pmatrix}
		1 \\ n_f
	\end{pmatrix}
\end{gather*}

Ich setze für $f = 4$ in die gegebenen Formeln ein.
\begin{gather*}
	\tens P
	=
	\left(
\begin{array}{cc}
 -0.7841& 2.8387\e{-16} \, \ii \\
 4.9059\e{-16}  \,\ii & -1.2754 \\
\end{array}
\right) \\
%
	\tens E
	=
	\begin{pmatrix}
		a \\ b
	\end{pmatrix}
	=
	\left(
	\begin{array}{c}
		+1.2947 -2.7876\e{-17} \, \ii \\
		-2.8629 +9.5331\e{-16} \, \ii \\
	\end{array}
	\right)
\end{gather*}

Daraus bestimme ich $r$:
\[
	r = \abs{\frac ab}^2 = 0.204527
\]

\subsection{alternierende Folge}

Mir fehlten Werte für $n_H$ und $n_L$, diese sind auch nicht aus der Rechnung
herausgefallen wir das $\lambda_0$ in der vorherigen Aufgabe.

Ausgehend von den Variablendeklarationen der vorherigen Teilaufgabe habe ich
einige geändert:
\begin{gather*}
	\delta_j := \pi \\
	%
	n_j := \begin{cases}
		n_H & \frac j2 \notin \Z \\
		n_L & \frac j2 \in \Z
	\end{cases}
\end{gather*}

Der Tensor $\tens M_j$ bleibt wie vorhin, nur dass das neue $n_j$ und
$\delta_j$ benutzt werden. Dann habe ich durch Ausprobieren so lange $\tens M_2
\tens M_1$ an den Tensor $\tens P$ gehängt, bis die Reflektivität bei $n_H =
1.5$ und $n_L = 1.1$ über $0.999$ gestiegen ist. Dafür musste ich $N = 30$
Schichten benutzen.

%%%%%%%%%%%%%%%%%%%%%%%%%%%%%%%%%%%%%%%%%%%%%%%%%%%%%%%%%%%%%%%%%%%%%%%%%%%%%%%
%                    Auflösungsvermögen eines Fernrohrs                     %
%%%%%%%%%%%%%%%%%%%%%%%%%%%%%%%%%%%%%%%%%%%%%%%%%%%%%%%%%%%%%%%%%%%%%%%%%%%%%%%

\section{Auflösungsvermögen eines Fernrohrs}
\label 2

Der Sehwinkel $\phi$ zwischen zwei Objekten muss größer sein als $\lambda/r$,
damit die Beugungsscheibchen sich nicht überdecken. Bei $r =
\unit{2.5}{\meter}$ und $\lambda = \unit{500}{\nano\meter}$ ist dies ein Winkel
von $\phi = \unit{200}{\nano\radian}$. Bei einer Entfernung von
$\unit{380}{\mega\meter}$ entspricht dies einem Abstand von
$\unit{76}{\meter}$.

Das Auge hat eine Auflösung von $\phi' = \unit{125}{\micro\radian}$, ist also
gut um einen Faktor $500$ schlechter. Die Objekte auf dem Mond müssten dann
schon $\unit{47.5}{\kilo\meter}$ auseinander sein.

%%%%%%%%%%%%%%%%%%%%%%%%%%%%%%%%%%%%%%%%%%%%%%%%%%%%%%%%%%%%%%%%%%%%%%%%%%%%%%%
%                            Fouriertransformation                            %
%%%%%%%%%%%%%%%%%%%%%%%%%%%%%%%%%%%%%%%%%%%%%%%%%%%%%%%%%%%%%%%%%%%%%%%%%%%%%%%

\section{Fourierreihenentwicklung und -transformation}
\label 3

\subsection{Fourierreihe}

Gegeben ist die Funktion $f(t) = \abs{\sin(t)}$. Es soll die Fourierreihe bestimmt werden.
\begin{align*}
	a_n
	&= \frac 1\pi \int_{-\pi}^\pi \dif t \, \abs{\sin(t)} \cos(nt) \\
	&= \frac 2\pi \int_0^\pi \dif t \, \sin(t) \cos(nt) \\
	&= \frac 2\pi \sbr{-\cos(t) \cos(nt)}_0^\pi - \frac 2\pi \int_0^\pi \dif t \, \cos(t) \sin(nt) n \\
	\intertext{%
		Ich wende wieder partielle Integration an, dabei ist der erste Term 0.
	}
	&= \frac 2\pi \del{(-1)^n+1} + \frac 2\pi \int_0^\pi \dif t \, \sin(t) \cos(nt) n^2 \\
	\intertext{%
		Jetzt habe ich etwas weiter oben und in der vorherigen Zeile das
		gleiche Integral stehen. Ich bringe beides auf eine Seite und erhalte:
	}
	\frac 2\pi\del{1 - n^2} \int_0^\pi \dif t \, \sin(t) \cos(nt) &= \frac 2\pi \del{(-1)^n+1} \\
							   \frac 2\pi \int_0^\pi \dif t \, \sin(t) \cos(nt) &= \frac{2}{\pi} \frac{(-1)^n+1}{1 - n^2} \\
	\intertext{%
		Dies ist gerade der gesuchte Koeffizient.
	}
							   a_n &= \frac{2}{\pi} \frac{(-1)^n+1}{1 - n^2}
\end{align*}

Die Bestimmung der $b_n$ geht schneller: Die Funktion $\abs{\sin(t)}$ ist
gerade, die Funktion $\sin(nt)$ ungerade. Somit ist das Integral über ein bei
0 zentriertes Intervall gerade 0.
\[
	b_n = \frac 1\pi \int_{-\pi}^\pi \dif t \, \abs{\sin(t)} \sin(nt) = 0
\]

Die Komponente $a_0$ ergibt sich aus $a_n$. Dies ist auch direkt der Gleichstromanteil.
\[
	a_0 = \frac 4\pi
\]

Die Fourierreihe ist:
\[
	f(t) = \frac 4 \pi + \sum_{n = 1}^\infty \frac{2}{\pi} \frac{(-1)^n+1}{1 - n^2} \cos(nt)
\]

\subsection{Puls}

Gegeben ist $f(t)$, es soll die Fouriertransformierte bestimmt werden:
\[
	f(t) = \begin{cases}
		\exp\del{-\gamma t} & t \geq 0 \\
									  0 & \text{sonst}
	\end{cases}
\]

Die Fouriertransformierte ist:
\begin{align*}
	f(\omega)
	&= \frac{1}{\sqrt{2 \pi}} \int_{-\infty}^\infty \dif t \, f(t) \exp\del{-\ii \omega t} \\
	&= \frac{1}{\sqrt{2 \pi}} \int_0^\infty \dif t \, \exp\del{-\gamma t} \exp\del{-\ii \omega t} \\
	&= \frac{1}{\sqrt{2 \pi}} \int_0^\infty \dif t \, \exp\del{-\del{\gamma + \ii \omega} t} \\
	&= \frac{1}{\sqrt{2 \pi}} \frac{1}{-\del{\gamma + \ii \omega}} \sbr{\exp\del{-\del{\gamma + \ii \omega} t}}_0^\infty \\
	&= \frac{1}{\sqrt{2 \pi}} \frac{1}{\gamma + \ii \omega} \\
	&= \frac{1}{\sqrt{2 \pi}} \frac{\gamma - \ii \omega}{\gamma^2 + \omega^2} \\
\end{align*}

\subsection{Amplitude und Phase}

Ich setze $\omega = \gamma$ ein. Der Betrag ist:
\[
	\frac{1}{\sqrt{\pi}} \frac{1}{2 \gamma} = \frac{1}{\sqrt{4 \pi} \gamma}
\]

Die Phase ist $- \pi / 4$.

Nun setze ich $\omega = 2 \gamma$ ein. Der Betrag ist:
\[
	\frac{1}{\sqrt{2 \pi}} \frac{\sqrt{5}}{5 \gamma} = \frac{1}{\sqrt{10 \pi} \gamma}
\]

Die Phase ist $\arctan\del{-2}$.

Das Amplitudenverhältnis ist:
\[
	\sqrt{\frac{10}{4}} = \sqrt{\frac 52}
\]

Und die Phasendifferenz:
\[
	\abs{\frac \pi4 - \arctan(2)}
\]

\subsection{Rechteckfunktion}

Gegeben ist eine Spektralfunktion $f(\omega)$:
\[
	f(\omega) = \begin{cases}
		\frac 1{\Deltaup \omega} & \abs{\omega - \omega_0} \leq \frac{\Deltaup \omega}2 \\
											  0 & \text{sonst}
	\end{cases}
\]

Die ursprüngliche Funktion ist:
\begin{align*}
	f(t)
	&= \frac{1}{\sqrt{2\pi}} \frac{1}{\Deltaup \omega} \int_{\omega_0 - \Deltaup \omega/2}^{\omega_0 + \Deltaup \omega/2} \dif \omega \, \exp\del{\ii \omega t} \\
	&= \frac{1}{\sqrt{2\pi}} \frac{1}{\ii t \Deltaup \omega} \sbr{\exp\del{\ii \omega t}}_{\omega_0 - \Deltaup \omega/2}^{\omega_0 + \Deltaup \omega/2} \\
	&= \frac{1}{\sqrt{2\pi}} \frac{1}{\ii t \Deltaup \omega} \exp\del{\ii \omega_0 t} \del{\exp\del{\ii \frac{\Deltaup \omega}2 t} - \exp\del{-\ii \frac{\Deltaup \omega}2 t}} \\
	&= \frac{1}{\sqrt{2\pi}} \frac{2}{\ii t \Deltaup \omega} \exp\del{\ii \omega_0 t} \sin\del{\frac{\Deltaup \omega}2 t}
\end{align*}

Für $\Deltaup \omega \to 0$ wird die Spektralfunktion zu einer $\delta$-Distribution. Diese enthält genau eine Frequenz, $\omega_0$, es kommt also genau eine Sinusschwingung heraus. Durch Einsetzen und die Regel von l'Hospital erhalte ich:
\[
	f(t) = \frac{1}{\sqrt{2 \pi} \ii} \exp\del{\ii \omega_0 t}
\]

Umgekehrt wird ein beliebig kurzer Lichtpuls eine Sinuswelle im Frequenzraum.

%%%%%%%%%%%%%%%%%%%%%%%%%%%%%%%%%%%%%%%%%%%%%%%%%%%%%%%%%%%%%%%%%%%%%%%%%%%%%%%
%                     Beugungsverluste in einem Resonator                     %
%%%%%%%%%%%%%%%%%%%%%%%%%%%%%%%%%%%%%%%%%%%%%%%%%%%%%%%%%%%%%%%%%%%%%%%%%%%%%%%

\section{Beugungsverluste in einem Resonator}
\label 4

\subsection{Fresnelzahl}

Der Radius der Zonen ist gegeben durch:
\[
	r^2 = \del{d + \frac{N \lambda}{2}}^2 - d^2
\]

Der maximale Radius ist der Spiegelradius, also $a$. Somit kann ich nach $N$ umstellen und erhalte:
\[
	n = \frac{a^2}{\lambda d}
\]

\subsection{relativer Begungsverlust}

\fehlt

\subsection{Lichtleistung}

\fehlt

%\bibliography{../../zentrale_BibTeX/Central}
%\bibliographystyle{plain}

\end{document}

% vim: spell spelllang=de
