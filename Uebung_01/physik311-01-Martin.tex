% Copyright © 2012 Martin Ueding <dev@martin-ueding.de>
%
\documentclass[11pt, ngerman]{article}

\usepackage[a4paper, left=3cm, right=2cm, top=2cm, bottom=2cm]{geometry}
\usepackage[activate]{pdfcprot}
\usepackage[cdot, squaren]{SIunits}
\usepackage[iso]{isodate}
\usepackage[parfill]{parskip}
\usepackage[T1]{fontenc}
\usepackage[utf8]{inputenc}
\usepackage{amsmath}
\usepackage{amssymb}
\usepackage{amsthm}
\usepackage{babel}
\usepackage{color}
\usepackage{commath}
\usepackage{epstopdf}
\usepackage{fancyhdr}
\usepackage{graphicx}
\usepackage{hyperref}
\usepackage{setspace}
\usepackage{tikz}

\usepackage[charter]{mathdesign}

\definecolor{darkblue}{rgb}{0,0,.5}
\definecolor{darkgreen}{rgb}{0,.5,0}

\hypersetup{
	breaklinks=false,
	citecolor=darkgreen,
	colorlinks=true,
	linkcolor=black,
	menucolor=black,
	urlcolor=darkblue,
}

\setlength{\columnsep}{2cm}

\DeclareMathOperator{\arcsinh}{arsinh}
\DeclareMathOperator{\arsinh}{arsinh}
\DeclareMathOperator{\asinh}{arsinh}
\DeclareMathOperator{\card}{card}
\DeclareMathOperator{\diam}{diam}

\newcommand{\dalambert}{\mathop{{}\Box}\nolimits}
\newcommand{\divergence}[1]{\inner{\vnabla}{#1}}
\newcommand{\ee}{\mathrm e}
\newcommand{\emesswert}{\del{\messwert \pm \messwert}}
\newcommand{\e}[1]{\cdot 10^{#1}}
\newcommand{\fehlt}{\textcolor{red}{Hier fehlen noch Inhalte.}}
\newcommand{\half}{\frac 12}
\newcommand{\ii}{\mathrm i}
\newcommand{\inner}[2]{\left\langle #1, #2 \right\rangle}
\newcommand{\laplace}{\mathop{{}\bigtriangleup}\nolimits}
\newcommand{\messwert}{\textcolor{blue}{\square}}
\newcommand{\punkte}{\textcolor{white}{xxxxx}}
\newcommand{\tens}[1]{\boldsymbol{#1}}
\newcommand{\vnabla}{\vec \nabla}
\renewcommand{\vec}[1]{\boldsymbol{#1}}

\newcommand{\themodul}{physik311}
\newcommand{\thegruppe}{Gruppe 3}
\newcommand{\theuebung}{1}

\pagestyle{fancy}

\fancyfoot[C]{\footnotesize{\thegruppe}}
\fancyfoot[L]{\footnotesize{Martin Ueding}}
\fancyfoot[R]{\footnotesize{Seite \thepage}}
\fancyhead[L]{\themodul{} -- Übung \theuebung}

\def\thesubsection{\thesection\alph{subsection}}

\title{\themodul{} -- Übung \theuebung \\ \vspace{0.5cm} \large{\thegruppe}}

\author{Martin Ueding \\ \small{\href{mailto:mu@uni-bonn.de}{mu@uni-bonn.de}}}

\begin{document}

\maketitle

%%%%%%%%%%%%%%%%%%%%%%%%%%%%%%%%%%%%%%%%%%%%%%%%%%%%%%%%%%%%%%%%%%%%%%%%%%%%%%%
%                                 Lichtdruck                                  %
%%%%%%%%%%%%%%%%%%%%%%%%%%%%%%%%%%%%%%%%%%%%%%%%%%%%%%%%%%%%%%%%%%%%%%%%%%%%%%%

\section{Lichtdruck}

\subsection{Druck auf schwarze Fläche}

Bei einer Wellenlänge $\lambda$ hat jedes Photon die Frequenz $f := \frac c
\lambda$ und somit Energie $W := h f$. Bei einer Intensität $I := P/A$ gibt es
pro Fläche $A$ eine Leistung $P$, was $\dot n := P / W$ Photonen entspricht.
Jedes Photon hat den Impuls $p := \frac{hf}{c}$, der bei der Absorption
komplett übertragen wird. Generell gilt $\dot p = F$. Somit gilt für den Druck:
\[
	\frac FA = \frac{\dot n h f}{A c} = \frac{P h f}{A W c} = \frac{P}{A c} = \frac{I}{c}
\]

Bei einer Intensität von $I = \unit{1}{\milli \watt \per \centi \meter
\squared}$ ist der Druck: $\unit{3.33 \e{-8}}{\pascal}$

\subsection{Beschleunigung auf Rubidiumatom}

Die Fläche, auf die das Licht wirkt, ist:
\[ \sigma = \frac{\lambda^2}{2 \pi} = \unit{9.6\e{-14}}{\meter \squared} \]

Somit ist die Beschleunigung auf die Masse von $m := \unit{87 \cdot 1.6\e{-27}}{\kilogram}$:
\[ a = \frac Fm = \frac{\sigma I}{cm} = \unit{2.32\e4}{\meter \per \second \squared} \]

%%%%%%%%%%%%%%%%%%%%%%%%%%%%%%%%%%%%%%%%%%%%%%%%%%%%%%%%%%%%%%%%%%%%%%%%%%%%%%%
%                            Fermat'sches Prinzip                             %
%%%%%%%%%%%%%%%%%%%%%%%%%%%%%%%%%%%%%%%%%%%%%%%%%%%%%%%%%%%%%%%%%%%%%%%%%%%%%%%

\section{fermatsches Prinzip}

Es soll das Brechungsgesetz hergeleitet werden:
\[ \frac{\sin\del{\theta_1}}{\sin\del{\theta_2}} = \frac{n_2}{n_1} \]

Das Licht breitet sich innerhalb des Mediums um $n$ langsamer aus, als im Vakuum.

\begin{figure}[h]
	\centering
	\begin{tikzpicture}
		\draw (-2, 0) -- (2, 0) node[right] {Mediumsgrenze};
		\draw[dashed] (0, -2) -- (0, 2) node[right] {optische Achse};
		\draw[thick, <-] (0, 0) -- (-1, 1) node[left] {$P_1$};
		\draw[thick, ->] (0, 0) -- ++(-30:2) node[right] {$P_2$};
		\draw (90:1) arc (90:135:1) node[label=90:$\theta_1$] {};
		\draw (-30:1) arc (330:270:1) node[below right] {$\theta_2$};
		\draw[|-|] (-1, 2.2) -- +(1, 0) node[midway, above] {$A$};
		\draw[|-|] (0, -2.2) -- +(1.7321, 0) node[midway, below] {$\del{1-A}$};
		\draw[|-|] (-1, -3) -- +(2.7321, 0) node[midway, below] {$1$};
		\draw[|-|] (-2.2, 0) -- +(0, 1) node[midway, left] {$1$};
		\draw[|-|] (-2.2, 0) -- +(0, -1) node[midway, left] {$1$};
	\end{tikzpicture}
	\caption{Skizze zur Lichtbrechung}
	\label{fig:Brechung}
\end{figure}

Angenommen, das Licht beginnt im Punkt $P_1(0, 1)$ und bewegt sich zum Punkt
$P_2(1, -1)$. Das Medium wird im Punkt $(A, 0)$ gewechselt. (Siehe Abbildung
\ref{fig:Brechung}.) Somit ist die Strecke, die das Licht zurücklegen muss:
\[ \sqrt{1 + A^2} + \sqrt{(1-A)^2 + 1^2} \]

Interessant ist allerdings die Zeit, die das Licht braucht:
\[ t := \frac 1{n_1 c} \sqrt{1 + A^2} + \frac 1{n_2 c} \sqrt{(1-A)^2 + 1} \]

Ich kann $A$ und $(1-A)$ auch durch den Sinus ausdrücken:
\[
	\sin\del{\theta_1} = \frac{A}{\sqrt{1 + A^2}}
	, \quad
	\sin\del{\theta_2} = \frac{1 - A}{\sqrt{1 + (1-A)^2}}
\]

Da eine extremale Zeit gefunden werden soll, leite ich nach $A$ ab und setze
gleich $0$:
%
\begin{align*}
\dpd t A &= 0 \\
\frac{2A}{2 n_1 c \sqrt{1+A^2}} - \frac{2(1-A)}{2 n_2 c \sqrt{(1-A^2)+1}} &= 0 \\
\frac{A}{n_1 \sqrt{1+A^2}} &= \frac{(1-A)}{n_2 \sqrt{(1-A^2)+1}} \\
\frac{n_1}{n_2} &= \frac{\frac{(1-A)}{\sqrt{(1-A^2)+1}}}{\frac{A}{\sqrt{1+A^2}}} \\
\intertext{Nun kann ich die Winkel einsetzen.}
\frac{n_1}{n_2} &= \frac{\sin\del{\theta_2}}{\sin\del{\theta_1}}
\end{align*}

Es ist anschaulich klar, dass dieses Extremum ein Minimum ist. Somit ist die
letzte Zeile die gesuchte Relation.

%%%%%%%%%%%%%%%%%%%%%%%%%%%%%%%%%%%%%%%%%%%%%%%%%%%%%%%%%%%%%%%%%%%%%%%%%%%%%%%
%                     zur Elektrodynamik an Grenzflächen                     %
%%%%%%%%%%%%%%%%%%%%%%%%%%%%%%%%%%%%%%%%%%%%%%%%%%%%%%%%%%%%%%%%%%%%%%%%%%%%%%%

\section{zur Elektrodynamik an Grenzflächen}

Es gilt, dass die Tangentialkomponenten von $\vec E$ und $\vec H$, und die
Normalkomponenten von $\vec B$ und $\vec D$ stetig sind.

Es soll wahrscheinlich gezeigt werden, dass keine Feldstärke verloren geht. In
der nächsten Aufgabe benötige ich die Relation $n_1 \del{E_e^2 - E_r^2}
\cos\del{\theta_i} = n_2 E_g^2 \cos\del{\theta_t}$, die ich hier wahrscheinlich
herleiten soll.

Die Energie $W$ einer Welle ist proportional zu $\sqrt{\epsilon} E^2$, wobei $n =
\sqrt{\epsilon}$ gilt. Da an der Grenzfläche keine freien Ladungen sind, die
Energie aufnehmen könnten, muss die Energie erhalten sein. Und zwar die Energie
des einfallenden Strahls abzüglich des reflektierten Strahls muss die Energie sein, die in das Medium eintritt. Somit muss gelten:
\[ W_e - W_r = W_d \]

Nun betrachte ich hier nur die Normalkomponente, so dass ich nur einen Anteil der Energie betrachten muss. Diesen Anteil liefert mir $\cos\del{\theta_i}$ und $\cos\del{\theta_t}$. Somit kann ich die Energien für die Wellen einsetzen:
\[
	\sqrt{\epsilon_1} \del{E_e^2 - E_r^2} \cos\del{\theta_i} = \sqrt{\epsilon_2} E_g^2 \cos\del{\theta_t}
\]

%%%%%%%%%%%%%%%%%%%%%%%%%%%%%%%%%%%%%%%%%%%%%%%%%%%%%%%%%%%%%%%%%%%%%%%%%%%%%%%
%                          Fresnel'sche Gleichungen                           %
%%%%%%%%%%%%%%%%%%%%%%%%%%%%%%%%%%%%%%%%%%%%%%%%%%%%%%%%%%%%%%%%%%%%%%%%%%%%%%%

\section{fresnelsche Gleichungen}

Es geht keine Energie an der Grenzfläche verloren. Also muss die Energie, die
in dem einfallenden ($e$), dem reflektierten ($r$) und gebrochene ($d$)
Strahl steckt, erhalten bleiben. Die Energie einer Welle ist proportional zu
$\sqrt{\epsilon} E^2 = n E^2$. Die Normalkomponente hängt von den Ein- und
Ausfallswinkeln $\theta_i$ beziehungsweise $\theta_t$ wie $\cos(\theta_i)$ und
$\cos(\theta_t)$ ab. Also gilt:
%
\begin{align*}
	n_1 \del{E_e^2 - E_r^2} \cos\del{\theta_i} &= n_2 E_g^2 \cos\del{\theta_t} \\
	%
	\intertext{
		Die Parallelkomponente des elektrischen Feldes ist (in der Summe)
		erhalten, es muss gelten: $E_e + E_r = E_g$. Nun teile ich die
		Gleichung durch die neue Relation und erhalte:
	}
	%
	n_1 \del{E_e - E_r} \cos\del{\theta_i} &= n_2 E_g \cos\del{\theta_t} \\
	%
	\intertext{
		Das Brechungsgesetz besagt $n_2 / n_1 = \sin(\alpha) / \sin(\beta)$.
		Dies kann ich umformen zu: $\sin(\beta) n_2 = n_1 \sin(\alpha)$. Ich
		teile die Gleichung durch die neue Relation und vertausche dann Sinus
		und Kosinus, um sie aus dem Nenner zu entfernen.
	}
	%
	\del{E_e - E_r} \sin\del{\theta_t} \cos\del{\theta_i} &= E_g
	\sin\del{\theta_i} \cos\del{\theta_t} \\
	%
	\intertext{
		Ich setze für $E_g$ die Summe $E_e + E_r$ ein.
	}
	%
	\del{E_e - E_r} \sin\del{\theta_t} \cos\del{\theta_i} &= \del{E_e + E_r}
	\sin\del{\theta_i} \cos\del{\theta_t} \\
	%
	\intertext{
		Das ganze forme ich jetzt nach $E_r$ um. Dabei benutze ich auch noch
		das Additionstheorem des Sinus.
	}
	%
	E_r &= - E_e \frac{\sin\del{\theta_i - \theta_t}}{\sin\del{\theta_i +
	\theta_t}}
	%
	\intertext{
		Dieses Ergebnis kann ich etwas früher einsetzen um $E_r$ zu eliminieren
		und erhalte eine Gleichung mit $E_e$ und $E_g$ für die Welle, die in
		das Medium eindringt:
	}
	%
	E_g &= E_e \frac{2 \sin\del{\theta_t} \cos\del{\theta_i}}{\sin\del{\theta_i +
	\theta_t}}
\end{align*}

Diese Herleitung stammt, bis hier, größtenteils aus \cite[Abschnitt
11.2.4]{meschede-gerthsen_24}.

Nun kann ich die beiden erhaltenen Formeln noch durch $E_e$ teilen und erhalte
die gesuchten Faktoren:
\[
	r_\perp = - \frac{\sin\del{\theta_i - \theta_t}}{\sin\del{\theta_i +
	\theta_t}}
	, \quad
	t_\perp = \frac{2 \sin\del{\theta_t} \cos\del{\theta_i}}{\sin\del{\theta_i +
	\theta_t}}
\]

\begin{figure}[ht]
	\centering
	\begin{minipage}[b]{0.3\linewidth}
		\centering
		\begin{tikzpicture}[scale=2, samples=100]
			\draw[dotted] (0, -1) grid (1.5, 1);
			\draw[thin, ->] (0, 0) -- ++(1.5, 0) node[right] {$\theta_i$};
			\draw[thin, ->] (0, -1) -- ++(0, 2);

			\draw[dashed] (0.02, 1) -- ++(0, -2);

			\foreach \x in {1}
			\draw (0, -1) ++(\x, 1pt) -- ++(0, -2pt) node[anchor=north] {$\unit\x\radian$};

			\foreach \y in {-1, 0, 1}
			\draw (1pt, \y) -- +(-2pt, 0) node[anchor=east] {$\y$};

			\draw (1, 0) node[above] {$t_\perp$};
			\draw (1, -1) node[above] {$r_\perp$};

			\clip (0, -1) rectangle (3, 1);

			\draw[thick, domain=0:1.5] plot (\x, {
				- sin(180 * (\x - 0.02) / 3.141) / sin(180 * (\x + 0.02) / 3.141)
			});
			\draw[thick, domain=0:1.5] plot (\x, {
				2 * sin(180 * 0.02 / 3.141) * cos(180 * \x / 3.141) / sin(180 * (\x + 0.02) / 3.141)
			});
		\end{tikzpicture}
		\caption{$\theta_t = \unit{0.02}\radian$}
	\end{minipage}
	\begin{minipage}[b]{0.3\linewidth}
		\centering
		\begin{tikzpicture}[scale=2, samples=100]
			\draw[dotted] (0, -1) grid (1.5, 1);
			\draw[thin, ->] (0, 0) -- ++(1.5, 0) node[right] {$\theta_i$};
			\draw[thin, ->] (0, -1) -- ++(0, 2);

			\draw[dashed] (0.10, 1) -- ++(0, -2);

			\foreach \x in {1}
			\draw (0, -1) ++(\x, 1pt) -- ++(0, -2pt) node[anchor=north] {$\unit\x\radian$};

			\foreach \y in {-1, 0, 1}
			\draw (1pt, \y) -- +(-2pt, 0) node[anchor=east] {$\y$};

			\draw (1, 0.3) node[above] {$t_\perp$};
			\draw (1, -0.7) node[above] {$r_\perp$};

			\clip (0, -1) rectangle (3, 1);

			\draw[thick, domain=0:1.5] plot (\x, {
				- sin(180 * (\x - 0.10) / 3.141) / sin(180 * (\x + 0.10) / 3.141)
			});
			\draw[thick, domain=0:1.5] plot (\x, {
				2 * sin(180 * 0.10 / 3.141) * cos(180 * \x / 3.141) / sin(180 * (\x + 0.10) / 3.141)
			});
		\end{tikzpicture}
		\caption{$\theta_t = \unit{0.10}\radian$}
	\end{minipage}
	\begin{minipage}[b]{0.3\linewidth}
		\centering
		\begin{tikzpicture}[scale=2, samples=100]
			\draw[dotted] (0, -1) grid (1.5, 1);
			\draw[thin, ->] (0, 0) -- ++(1.5, 0) node[right] {$\theta_i$};
			\draw[thin, ->] (0, -1) -- ++(0, 2);

			\draw[dashed] (0.50, 1) -- ++(0, -2);

			\foreach \x in {1}
			\draw (0, -1) ++(\x, 1pt) -- ++(0, -2pt) node[anchor=north] {$\unit\x\radian$};

			\foreach \y in {-1, 0, 1}
			\draw (1pt, \y) -- +(-2pt, 0) node[anchor=east] {$\y$};

			\draw (1, 0.1) node[above] {$t_\perp$};
			\draw (1, -0.9) node[above] {$r_\perp$};

			\clip (0, -1) rectangle (3, 1);

			\draw[thick, domain=0:1.5] plot (\x, {
				- sin(180 * (\x - 0.50) / 3.141) / sin(180 * (\x + 0.50) / 3.141)
			});
			\draw[thick, domain=0:1.5] plot (\x, {
				2 * sin(180 * 0.50 / 3.141) * cos(180 * \x / 3.141) / sin(180 * (\x + 0.50) / 3.141)
			});
		\end{tikzpicture}
		\caption{$\theta_t = \unit{0.50}\radian$}
	\end{minipage}
\end{figure}

Der Brewsterwinkel ist erreicht, wenn $r_\perp = 0$ ist. Dies ist der Schnitt
der $r_\perp$ Kurve mit der $\theta_i$-Achse. Der Grenzwinkel der Totalreflexion
ist erreicht, wenn $t_\perp = 1$ ist. Der Winkel ist in allen Fällen $\theta_t$
und ist mit einer gestrichelten Linie eingezeichnet.

\bibliography{../../zentrale_BibTeX/Central}
\bibliographystyle{plain}

\end{document}

% vim: spell spelllang=de
