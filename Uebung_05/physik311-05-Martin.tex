% Copyright © 2012 Martin Ueding <dev@martin-ueding.de>
%
\documentclass[11pt, ngerman, fleqn]{article}

\usepackage[a4paper, left=3cm, right=2cm, top=2cm, bottom=2cm]{geometry}
\usepackage[activate]{pdfcprot}
\usepackage[thinspace, squaren]{SIunits}
\usepackage[iso]{isodate}
\usepackage[parfill]{parskip}
\usepackage[T1]{fontenc}
\usepackage[utf8]{inputenc}
\usepackage{amsmath}
\usepackage{amsthm}
\usepackage{babel}
\usepackage{color}
\usepackage{lastpage}
\usepackage{commath}
\usepackage{epstopdf}
\usepackage{fancyhdr}
\usepackage{graphicx}
\usepackage{hyperref}
\usepackage{setspace}
\usepackage{tikz}

\usepackage[charter, greekuppercase=italicized]{mathdesign}

\usetikzlibrary{arrows}
\usetikzlibrary{intersections}

\definecolor{darkblue}{rgb}{0,0,.5}
\definecolor{darkgreen}{rgb}{0,.5,0}

\hypersetup{
	breaklinks=false,
	citecolor=darkgreen,
	colorlinks=true,
	linkcolor=black,
	menucolor=black,
	urlcolor=darkblue,
}

\setlength{\columnsep}{1.5cm}

\DeclareMathOperator{\arcsinh}{arsinh}
\DeclareMathOperator{\arsinh}{arsinh}
\DeclareMathOperator{\asinh}{arsinh}
\DeclareMathOperator{\card}{card}
\DeclareMathOperator{\diam}{diam}

\newcommand{\dalambert}{\mathop{{}\Box}\nolimits}
\newcommand{\divergence}[1]{\inner{\vnabla}{#1}}
\newcommand{\ee}{\mathrm e}
\newcommand{\emesswert}{\del{\messwert \pm \messwert}}
\newcommand{\e}[1]{\cdot 10^{#1}}
\newcommand{\fehlt}{\textcolor{red}{Hier fehlen noch Inhalte.}}
\newcommand{\half}{\frac 12}
\newcommand{\ii}{\mathrm i}
\newcommand{\inner}[2]{\left\langle #1, #2 \right\rangle}
\newcommand{\laplace}{\mathop{{}\bigtriangleup}\nolimits}
\newcommand{\messwert}{\textcolor{blue}{\square}}
\newcommand{\punkte}{\textcolor{white}{xxxxxxx}}
\newcommand{\tens}[1]{\boldsymbol{\mathsf{#1}}}
\newcommand{\vnabla}{\vec \nabla}
\renewcommand{\vec}[1]{\boldsymbol{#1}}

\newcommand{\themodul}{physik311}
\newcommand{\thegruppe}{Gruppe 3 -- Matthias Rehberger}
\newcommand{\theuebung}{5}

\pagestyle{fancy}

\fancyfoot[C]{\footnotesize{\thegruppe}}
\fancyfoot[L]{\footnotesize{Martin Ueding}}
\fancyfoot[R]{\footnotesize{Seite \thepage\ / \pageref{LastPage}}}
\fancyhead[L]{\themodul{} -- Übung \theuebung}

\setcounter{section}{16}

\def\thesubsection{\thesection\alph{subsection}}

\title{\themodul{} -- Übung \theuebung \\ \vspace{0.5cm} \large{\thegruppe}}

\author{Martin Ueding \\ \small{\href{mailto:mu@uni-bonn.de}{mu@uni-bonn.de}}}

\begin{document}

\maketitle

\begin{table}[h]
	\centering
	\begin{tabular}{l|c|c|c|c}
		Aufgabe & \ref{1} & \ref{2} & \ref{3} & $\sum$   \\
		\hline
		Punkte & \punkte & \punkte & \punkte & \punkte
	\end{tabular}
\end{table}

%%%%%%%%%%%%%%%%%%%%%%%%%%%%%%%%%%%%%%%%%%%%%%%%%%%%%%%%%%%%%%%%%%%%%%%%%%%%%%%
%                                ABCD-Matrizen                                %
%%%%%%%%%%%%%%%%%%%%%%%%%%%%%%%%%%%%%%%%%%%%%%%%%%%%%%%%%%%%%%%%%%%%%%%%%%%%%%%

\section{ABCD-Matrizen}
\label 1

\subsection{Zeichnung}

Eine Zeichnung ist in Abbildung \ref{fig:linsen}.

\subsection{ABCD-Matrix des Systems}

\begin{figure}
	\centering
	\begin{tikzpicture}[scale=2]
		\draw (-1, 0) -- (4.5, 0) node[right] {optische Achse};

		\draw[|-|] (0, -1.1) -- ++(3.5, 0) node[below, midway] {$\unit{35}{\centi \meter}$};

		\draw[|-|] (0, 1.1) -- ++(1.0, 0) node[above, midway] {$f_1 = \unit{10}{\centi \meter}$};
		\draw[|-|] (0, 1.1) -- ++(-1.0, 0) node[above, midway] {$f_1 = \unit{10}{\centi \meter}$};

		\draw[|-|] (3.5, 1.1) -- ++(2.0, 0) node[above, midway] {$f_2 = \unit{20}{\centi \meter}$};
		\draw[|-|] (3.5, 1.1) -- ++(-2.0, 0) node[above, midway] {$f_2 = \unit{20}{\centi \meter}$};

		\pgfpathmoveto{\pgfpoint{0cm}{-1cm}}
		\pgfpatharcto{3cm}{3cm}{0}{0}{0}{\pgfpoint{0cm}{1cm}}
		\pgfusepath{draw}

		\pgfpathmoveto{\pgfpoint{0cm}{-1cm}}
		\pgfpatharcto{3cm}{3cm}{0}{0}{1}{\pgfpoint{0cm}{1cm}}
		\pgfusepath{draw}

		\pgfpathmoveto{\pgfpoint{3.5cm}{-1cm}}
		\pgfpatharcto{5cm}{5cm}{0}{0}{0}{\pgfpoint{3.5cm}{1cm}}
		\pgfusepath{draw}

		\pgfpathmoveto{\pgfpoint{3.5cm}{-1cm}}
		\pgfpatharcto{5cm}{5cm}{0}{0}{1}{\pgfpoint{3.5cm}{1cm}}
		\pgfusepath{draw}
	\end{tikzpicture}
	\caption{Skizze zu Aufgabe \ref 1}
	\label{fig:linsen}
\end{figure}

Die Matrizen der beiden Linsen und des Weges in der Mitte sind:
\[
	\tens A = \begin{pmatrix}
		1 & 0 \\ - \unit{10}{\reciprocal \meter} & 1
	\end{pmatrix}
	,\quad
	\tens B = \begin{pmatrix}
		1 & \unit{0.35}\meter \\ 0 & 1
	\end{pmatrix}
	,\quad
	\tens C = \begin{pmatrix}
		1 & 0 \\ - \unit{5}{\reciprocal \meter} & 1
	\end{pmatrix}
\]

Die Matrix $\tens D$ für das Gesamtsystem berechnet sich durch $D^a{}_d :=
A^a{}_b B^b{}_c C^c{}_d$ (mit Summenkonvention) (Abbildung \ref{fig:penrose})
zu:
\[
	\tens D = \begin{pmatrix}
		-0.75 & \unit{0.35} \meter \\ \unit{2.5}{\reciprocal \meter} & -2.5
	\end{pmatrix}
\]

\begin{figure}
	\centering
	\begin{tikzpicture}
		\node[shape=rectangle, draw] (A) at (0, 1) {$\tens A$};
		\node[shape=rectangle, draw] (B) at (0, 0) {$\tens B$};
		\node[shape=rectangle, draw] (C) at (0, -1) {$\tens C$};
		\node[shape=rectangle, draw] (D) at (-2, 0) {$\tens D$};
		\node (e) at (-1, 0) {$=$};


		\draw (A) -- (B) -- (C);

		\draw (A) -- +(0, 0.5);
		\draw (C) -- +(0, -0.5);

		\draw (D) -- +(0, 0.5);
		\draw (D) -- +(0, -0.5);
	\end{tikzpicture}
	\caption{penrosesche graphische Notation zur Tensorkontraktion}
	\label{fig:penrose}
\end{figure}

Die Determinante $\det\del{\tens D}$ ist 1.

\subsection{Lichtstrahl}

Ein Lichtstrahl ist gegeben durch:
\[
	\vec l = \begin{pmatrix}
		\unit{1}{\milli\meter} \\ 0
	\end{pmatrix}
\]

Die Position nach der Linsenanordnung, $\vec n$, wird durch $n^a = D^a{}_b l^b$
gegeben:
\[
	\vec n = \begin{pmatrix}
		\unit{-0.75}{\milli\meter} \\ \unit{2.5}{\milli\radian}
	\end{pmatrix}
\]

Und die Position noch einen Meter später, $\vec m$, wird mit der geraden
Ausbreitungsmatrix $\tens E$ bestimmt durch $m^a = E^a{}_b n^b$:
\[
	\tens E = \begin{pmatrix}
		1 & \unit{1}{\meter} \\ 0 & 1
	\end{pmatrix}
	, \quad
	\vec m = \begin{pmatrix}
		\unit{1.75}{\milli\meter} \\
		\unit{2.5}{\milli\rad}
	\end{pmatrix}
\]

Der Strahl hat also nach der Linsenanordnung und nach einem Meter den Abstand
$\unit{0.75}{\milli\meter}$ be\-zieh\-ungs\-weise $\unit{1.75}{\milli\meter}$ von der
Achse.

%%%%%%%%%%%%%%%%%%%%%%%%%%%%%%%%%%%%%%%%%%%%%%%%%%%%%%%%%%%%%%%%%%%%%%%%%%%%%%%
%                       konfokaler optischer Resonator                        %
%%%%%%%%%%%%%%%%%%%%%%%%%%%%%%%%%%%%%%%%%%%%%%%%%%%%%%%%%%%%%%%%%%%%%%%%%%%%%%%

\section{konfokaler optischer Resonator}
\label 2

Die Matrix für den Spiegel und Distanz sind:
\[
	\tens S = \begin{pmatrix}
		1 & 0 \\ - \frac 2r & 1
	\end{pmatrix}
	, \quad
	\tens D = \begin{pmatrix}
		1 & d \\ 0 & 1
	\end{pmatrix}
\]

Zusammen ergeben Sie die Matrix $\tens R$ für einen kompletten Zyklus im
Resonator (\( R^a{}_e = S^a{}_b D^b{}_c S^c{}_d D^d{}_e \)):
\[
	\tens R = \begin{pmatrix}
		\frac{r-2d}r & \frac{2d(r-d)}{r} \\
		\frac{4(d-r)}{r^2} & \frac{4d^2 - 6dr + r^2}{r^2}
	\end{pmatrix}
\]

Dies entspricht der Matrix, die auf dem Aufgabenblatt angegeben ist.

Für den Fall $r = d$ vereinfacht sich die Matrix zur negativen Einheitsmatrix.
Das Quadrat davon ist die Einheitsmatrix.

%%%%%%%%%%%%%%%%%%%%%%%%%%%%%%%%%%%%%%%%%%%%%%%%%%%%%%%%%%%%%%%%%%%%%%%%%%%%%%%
%                        achromatisches Linsendublett                         %
%%%%%%%%%%%%%%%%%%%%%%%%%%%%%%%%%%%%%%%%%%%%%%%%%%%%%%%%%%%%%%%%%%%%%%%%%%%%%%%

\section{achromatisches Linsendublett}
\label 3

\subsection{Herleitung}

Ich schreibe $f_x := f\del{\lambda_x}$. Zur Herleitung beginne ich mit der
Definition von $A$:
\begin{align*}
	A &= f_r - f_b \\
	A &= \frac{f_r - f_b}{f_g^2} f_g^2 \\
	\intertext{Ich benutze die zweite Beziehung.}
	A &= \frac{f_r - f_b}{f_r f_g} f_g^2 \\
	A &= \del{\frac{1}{f_b} - \frac{1}{f_b}} f_g^2 \\
	A &= \frac{\frac{1}{f_b} - \frac{1}{f_b}}{\frac{1}{f_g}} f_g \\
	A &= \frac{\frac{k}{f_b} - \frac{k}{f_b}}{\frac{k}{f_g}} f_g \\
	\intertext{Ich benutze die erste Beziehung.}
	A &= \frac{(n_b-1) - (1-n_r)}{n_g-1} f_g \\
	A &= \frac{n_b - n_r}{n_g-1} f_g \\
	\intertext{Ich setze $B$ ein.}
	A &= B f_g \\
\end{align*}

\subsection{Verhältnis}

Es soll für das die chromatische Aberration $A$ des Gesamtsystems gelten: $A =
0$. Die Brechkräfte der einzelnen Linsen addieren sich auf, so dass für die
Summen gelten muss:
\begin{align*}
	f_r - f_b &= 0 \\
	\intertext{Diese Brennweiten kann ich jetzt als Summe der einzelnen
	Brennweiten ausdrücken:}
	\frac{1}{\frac{1}{f_{1,r}} + \frac{1}{f_{2,r}}} - \frac{1}{\frac{1}{f_{1,b}} + \frac{1}{f_{2,b}}} &= 0 \\
	\frac{f_{1,r} f_{2,r}}{f_{1,r} + f_{2,r}} - \frac{f_{1,b} f_{2,b}}{f_{1,b} + f_{2,b}} &= 0 \\
	\intertext{Jetzt forme ich das ganze noch etwas weiter um. Nach fünf
	Schritten erhalte ich:}
	\frac{f_{1,r} f_{1,b}}{f_{1,r} + f_{1,b}} - \frac{f_{2,r} f_{2,b}}{f_{2,r} + f_{2,b}} &= 0 \\
	\intertext{An dieser Stelle benutze ich die zweite Beziehung aus der
	vorherigen Aufgabe sowie die Definition von $A$.}
	\frac{f_{1,g}^2}{A_1} + \frac{f_{2,g}^2}{A_2} &= 0 \\
	\intertext{Nun setze ich $A = B f_g$ ein und kürze das $f_g$.}
	\frac{f_{1,g}}{B_1} + \frac{f_{2,g}}{B_2} &= 0 \\
	\intertext{Umgestellt erhalte ich die gesuchte Beziehung.}
	\frac{f_{1,g}}{f_{2,g}} &= - \frac{B_1}{B_2} \\
\end{align*}

Hierbei zählt die Brennweite der konkaven Linse als negativ. Würde man hier
Beträge benutzen, entfiele das Minus.

\subsection{konkrete Werte}

Zuerst bestimme ich $B_1$ und $B_2$ aus den Daten:
\[
	B_1 = 0.0154739
	,\quad
	B_2 = 0.0274194
\]

Die Brennweite einer gekrümmten Fläche
ist\footnote{\url{https://de.wikipedia.org/wiki/Brennweite\#Brechende_Fl.C3.A4che}}:
\[
	f' = r \frac{n'}{n'-n}
\]

Die Gesamtbrennweite des Systems ist:
\[
	\frac 1f = \frac{n_1-1}{r_1 n_1} + \frac{n_2-n_1}{r_2 n_2} + \underbrace{\frac{1-n_2}{r_3}}_{\to 0}
\]

Um eine zweite Gleichung für dieses System zu erhalten, muss ich das ganze für rotes und blaues Licht betrachten:
\begin{align*}
	\unit{4}{\reciprocal\meter} &= \frac{n_{1,r}-1}{r_1 n_{1,r}} + \frac{n_{2,r}-n_{1,r}}{r_2 n_{2,r}} \\
	\unit{4}{\reciprocal\meter} &= \frac{n_{1,b}-1}{r_1 n_{1,b}} + \frac{n_{2,b}-n_{1,b}}{r_2 n_{2,b}} \\
\end{align*}

Dieses Gleichungssystem habe ich in Mathematica nach $r_1$ und $r_2$ lösen lassen. Als Ergebnis erhalte ich:
\[
	r_1 = \unit{0.0946487}\meter
	\quad\wedge\quad
	r_2 = \unit{0.151399}\meter
\]

Wenn ich allerdings grün und rot benutze, erhalte ich:
\[
	r_1 = \unit{0.0949488}\meter
	\quad\wedge\quad
	r_2 = \unit{0.154827}\meter
\]

Eigentlich sollte das gleiche herauskommen. Außerdem würde ich erwarten, dass
der $r_2$ negativ ist, so wie in der Zeichnung dargestellt.

%\bibliography{../../zentrale_BibTeX/Central}
%\bibliographystyle{plain}

\end{document}

% vim: spell spelllang=de
