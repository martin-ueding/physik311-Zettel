% Copyright © 2012 Martin Ueding <dev@martin-ueding.de>
%
\input{header.tex}

\usepackage{fancyhdr}
\usepackage{tikz}

\newcommand{\themodul}{physik311}
\newcommand{\thegruppe}{Gruppe 3 -- Matthias Rehberger}
\newcommand{\theuebung}{9}

\pagestyle{fancy}

\fancyfoot[C]{\footnotesize{\thegruppe}}
\fancyfoot[L]{\footnotesize{Martin Ueding}}
\fancyfoot[R]{\footnotesize{Seite \thepage\ / \pageref{LastPage}}}
\fancyhead[L]{\themodul{} -- Übung \theuebung}

\setcounter{section}{29}

\def\thesubsection{\thesection\alph{subsection}}

\title{\themodul{} -- Übung \theuebung \\ \vspace{0.5cm} \large{\thegruppe}}

\author{Martin Ueding \\ \small{\href{mailto:mu@uni-bonn.de}{mu@uni-bonn.de}}}

\begin{document}

\maketitle

\begin{table}[h]
	\centering
	\begin{tabular}{l|c|c|c|c}
		Aufgabe & \ref{1} & \ref{2} & \ref{3} & $\sum$   \\
		\hline
		Punkte & \punkte & \punkte & \punkte & \punkte
	\end{tabular}
\end{table}

Es gibt zwei Aufgaben 30, die zweite Aufgabe 30 sollte wohl \ref 2, und die
Aufgabe 31 eigentlich \ref 3 sein.

%%%%%%%%%%%%%%%%%%%%%%%%%%%%%%%%%%%%%%%%%%%%%%%%%%%%%%%%%%%%%%%%%%%%%%%%%%%%%%%
%                          Beugung am Mehrfachspalt                           %
%%%%%%%%%%%%%%%%%%%%%%%%%%%%%%%%%%%%%%%%%%%%%%%%%%%%%%%%%%%%%%%%%%%%%%%%%%%%%%%

\section{Beugung am Mehrfachspalt}
\label 1

\subsection{Verlauf der Funktion}

Der Verlauf der Funktion ist in Abbildung \ref{fig:30a1} und \ref{fig:30a2}
gezeigt. Für $b \ll a N$ kann der erste Term genähert werden und wird $1$. Dies
sollte unendlich dünnen Spalten entsprechen. Für den zweiten Fall sind die
Beugungsbilder der einzelnen Spalte zu erkennen, als auch die Maxima durch den
Doppelspalt.

\begin{figure}
	\centering
	\begin{minipage}{0.45\textwidth}
		\centering
		\includegraphics[width=\textwidth]{30a1.pdf}
		\caption{$k = 300, b = 0.1, a = 3, N = 1$}
		\label{fig:30a1}
	\end{minipage}
	\hspace{0.08\textwidth}
	\begin{minipage}{0.45\textwidth}
		\centering
		\includegraphics[width=\textwidth]{30a2.pdf}
		\caption{$k = 5, b = 1, a = 4, N = 6$}
		\label{fig:30a2}
	\end{minipage}
\end{figure}

\subsection{Maxima des Gitters}

Die Beugung an den einzelnen Spalten erzeugt nur ein Hauptmaximum, weitere
Hauptmaxima liegen aufgrund der unendlichen Spaltzahl im Einzelspalt unendlich
weit auseinander.

Es bleiben also die Maxima des Gitters selbst als Maxima. Ein Maximum tritt immer dann ein, wenn im letzten Term Zähler und Nenner gegen 0 gehen. Es muss also gelten:
\[
	\frac{ka}{2} \sin \Theta = n \pi
	\iff
	\Theta = \arcsin\del{\frac{2 n \pi}{ka}}
\]

Die Spaltzahl $N = 1000$ ist hier noch egal, die Spaltbreite $b = \unit{1500}{\nano\meter}$ auch. Der Abstand $a = \unit{6000}{\nano\meter}$ sowie die Wellenlänge $\lambda = \unit{500}{\nano\meter}$ sind entscheidend. Die Wellenzahl ist:
\[
	k = \frac{2 \pi}{\lambda}
\]

Dies setze ich ein:
\[
	\Theta = \arcsin\del{\frac{n \lambda}{a}}
\]

Dann setze ich $n = 0, 1, 2$ ein und erhalte:
\begin{align*}
	\Theta_0 &= \unit{0}{\rad} \\
	\Theta_1 &= \unit{0.0834301}{\rad} \\
	\Theta_2 &= \unit{0.167448}{\rad} \\
\end{align*}

\subsection{Minima}

Für die Minima ist die Spaltanzahl jetzt wichtig. Minima entstehen, wenn einer
der Zähler 0 wird, der Nenner allerdings nicht. Dabei kann nur der zweite
Zähler alleine 0 werden. Denn der erste Zähler ist genau der zweite Nenner.
\[
	N \frac{ka}2 \sin \Theta = n \pi
	\quad
	\wedge
	\quad
	\frac{ka}2 \sin \Theta \neq m \pi
\]

\[
	\Theta = \arcsin\del{\frac{n \lambda}{Na}}
	\quad
	\wedge
	\quad
	\frac nN \notin \mathbb Z
\]

Gesucht sind jetzt $n = 1, 1000 \pm 1, 2000 \pm 1$.
\begin{align*}
	\Theta_1' &= \unit{0}{\rad} \\
\Theta_{999}' &= \unit{0.0835137}{\rad} \\
\Theta_{1001}' &= \unit{0.0833465}{\rad} \\
\Theta_{1999}' &= \unit{0.167364}{\rad} \\
\Theta_{2001}' &= \unit{0.167533}{\rad} \\
\end{align*}

\subsection{Wellenlänge}

In das 0. Maximum wird immer gebrochen, da ist wie Wellenlänge irrelevant. Für das erste Maximum setze ich an:
\[
	\Theta = \arcsin\del{\frac{\lambda'_1}{a}} = \unit{0.0833465}{\rad}
\]

Ich erhalte $\lambda'_1 = \unit{500.5}{\nano\meter}$. Analog erhalte ich für das zweite Maximum $\lambda'_2 = \unit{500.251}{\nano\meter}$.

Somit sind die spektralen Auflösungen:
\begin{gather*}
	\frac{\Deltaup \lambda}{\lambda} = 0.001 \\
	\frac{\Deltaup \lambda}{\lambda} = 0.0005021
\end{gather*}

Es gilt also:
\[
	\frac{\Deltaup \lambda}{\lambda} = \frac{1}{n N}
\]

\subsection{Reflexionsgitter}

Bei einem Reflexionsgitter sind die Reflexstreifen ähnlich wie beim
durchlassenden Gitter wieder Lichtquellen, die Kugelwellen ausstrahlen, da sie
so klein sind. Die Geometrie hat dann zwar einen Knick, ist ansonsten ähnlich,
sogar kongruent.

%%%%%%%%%%%%%%%%%%%%%%%%%%%%%%%%%%%%%%%%%%%%%%%%%%%%%%%%%%%%%%%%%%%%%%%%%%%%%%%
%                                Stufengitter                                 %
%%%%%%%%%%%%%%%%%%%%%%%%%%%%%%%%%%%%%%%%%%%%%%%%%%%%%%%%%%%%%%%%%%%%%%%%%%%%%%%

\section{Stufengitter}
\label 2

\subsection{Lage der Extremstellen}

Aus der Zeichnung in Abbildung \ref{fig:stufengitter} kann ich folgende
Relation ablesen:
\[
	\frac d{\frac{\Deltaup s}2} = \cos\del{\frac\Theta 2}
\]

Dies stelle ich nach $\Deltaup s$ um und setze es gleich $n \lambda$ für die
konstruktive Interferenz:
\[
	n \lambda = 2 d \sec\del{\frac \Theta 2}
	\iff
	\Theta_\text{max} = 2 \arccos\del{\frac{2d}{n\lambda}}
\]

Für das Minimum ähnlich:
\[
	\del{n + \half} \lambda = 2 d \sec\del{\frac \Theta 2}
	\iff
	\Theta_\text{min} = 2 \arccos\del{\frac{2d}{\del{n + \half}\lambda}}
\]


\begin{figure}
	\centering
	\begin{tikzpicture}
		\draw (-5, 1) -- (0, 1) -- (0, -1) node[midway, left] {$d$} -- (5, -1);
		\draw[->] (-1, 1) -- ++(60:4);
		\draw[<-] (-1, 1) -- ++(120:4);
		\draw[->] (2, -1) -- ++(60:6);
		\draw[<-] (2, -1) -- ++(120:6);
		\draw[dashed] (2, -1) -- ++(0, 2) -- ++(1, 0);
		\draw (2, -1) ++(60:1) arc (60:90:1) node[above right] {$\frac \Theta 2$};
		\draw[|-|] (2, -1) ++(-30:0.3) -- ++(60:2.31) node[midway, right] {$\half \Deltaup s$};
	\end{tikzpicture}
	\caption{Skizze zum Stufengitter}
	\label{fig:stufengitter}
\end{figure}

Der wichtige Trick ist nun, dass bei das nullte Maximum nicht bei $n = 0$
liegen kann. Dieses sollte bei $\Theta_\text{max} = \unit 0 \radian$ liegen,
somit muss das Argument des $\arccos$ gleich $1$ sein:
\[
	1 = \frac{2d}{n \lambda}
	\implies
	n = 1000
\]

Also müssen die Ordnungszahlen $k = 0, 1, \ldots$ umgerechnet werden mit $n =
1000 + k$.

\subsection{Einsetzen, Nachbarn}

Das 500. Maximum ist also dann das mit der Nummer $n = 1500$. Dies setze ich in
$\Theta_\text{max}$ ein und erhalte:
\[
	\Theta_\text{max} = \unit{1.68214}\radian
\]

Die Minima erhalte durch Einsetzen von $n = 1499$ und $n = 1500$ in die Formel
für die Minima:
\[
	\Theta_\text{min} = \unit{1.68273}\radian
	,\quad
	\Theta_\text{min}' = \unit{1.68154}\radian
\]

Ich setze das erste Minimum in die Formel für das Maximum ein und löse das
ganze nach $\lambda$ auf:
\[
	\cos\del{\frac{\Theta_\text{min}}2} = \frac{2d}{n \lambda'}
	\implies
	\lambda' = \frac{3001}{4000000} \frac 1n
\]

Dort setze ich das gegebene $\lambda$ der Aufgabenstellung ein und erhalte für
$n$:
\[
	n = 1500.5
\]

Da nur ganzzahlige $n$ erlaubt sind, muss ich runden. Für $n = 1501$ ist
$\lambda' = \unit{499.833}{\nano\meter}$ und der Abstand $\abs{\lambda' -
\lambda} = \unit{16.656}{\nano\meter}$ minimiert.

\subsection{spektrale Auflösung}

Damit kann ich die spektrale Auflösung bestimmen:
\[
	\frac{\Deltaup \lambda}{\lambda}
	= \frac{16.656}{500}
	= 0.033312
\]

%%%%%%%%%%%%%%%%%%%%%%%%%%%%%%%%%%%%%%%%%%%%%%%%%%%%%%%%%%%%%%%%%%%%%%%%%%%%%%%
%                     michelsonsches Sterninterferometer                      %
%%%%%%%%%%%%%%%%%%%%%%%%%%%%%%%%%%%%%%%%%%%%%%%%%%%%%%%%%%%%%%%%%%%%%%%%%%%%%%%

\section{michelsonsches Sterninterferometer}
\label 3

\subsection{Interferenzmuster für einen Stern}

Der erste Stern liegt bei $\phi = 0$, so dass sich durch den vorderen Aufbau
kein Gangunterschied ergibt. Vom vorherigen Zettel übernehme ich die Formel für
die Intensität zweier Wellen im Doppelspalt. Dabei kann ich anscheinend
$\theta$ als klein annehmen, wie in der Zeichnung gezeigt.
\[
	I =
	I_0 \sin^2\del{\frac{b 2 \pi}{\lambda} \theta} \csc^2\del{\frac{b \pi}{\lambda} \theta}
\]


\subsection{Warum interferiert das Licht der Sterne nicht miteinander?}

Das Licht der beiden Sterne interferiert nicht untereinander, weil das Licht
kohärent sind, die Sterne sind thermische Strahler.

\subsection{Beide Sterne}

Bei $h \to 0$ entspricht der Versuch dem normalen Doppelspalt, daher ist nur
die Wellenlänge $\lambda$ entscheidend. Da aus den Spalten Kugelwellen
ausgesendet werden, spielt die ursprüngliche Richtung keine Rolle.

Für $h \neq 0$ bekommt der zweite Stern einen zusätzlichen Gangunterschied von
$h \phi$ hinzu. Wenn nun bei $h = \unit 5 \meter$ kein Interferenzmuster mehr
zu erkennen ist, dann liegen die Maxima des einen Sterns in den Minima des
anderen Sterns. Dazu muss $h \phi k = \pi$, also $h \phi 2 \pi / \lambda = \pi$ gelten. Ich stelle um und erhalte:
\[
	\phi = \frac{\lambda}{2 h} = \unit{50}{\nano\rad}
\]

%\bibliography{../../zentrale_BibTeX/Central}
%\bibliographystyle{plain}

\end{document}

% vim: spell spelllang=de
